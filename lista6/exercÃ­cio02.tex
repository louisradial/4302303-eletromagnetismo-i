\begin{exercício}{Condições de contorno para meios lineares}{exercício2}
    Na interface entre dois meios dielétricos, o campo elétrico deve necessariamente ser consistente com as condições de contorno que você resumiu no problema anterior. Assim, considere uma interface genérica entre dois meios \emph{lineares}, como mostra a figura abaixo. As constantes \(\epsilon_1\) e \(\epsilon_2\) correspondem às permissividades elétricas de cada um dos meios e os campos elétricos estão indicados.

    Prove que \(\epsilon_2 \tan{\theta_1} = \epsilon_1 \tanh{\theta_2}\).
\end{exercício}
\begin{proof}[Resolução]

\end{proof}
