\begin{exercício}{}{exercício6}
    Considere uma esfera (maciça) de raio \(R\) constituída de um material dielétrico com constante dielétrica \(k\). No centro desse objeto é colocado um dipolo (ideal) \(\vetor{p}\). Encontre o potencial elétrico em todo o espaço. Considere vácuo para \(r > R\). Calcule a densidade de carga superficial de polarização \(\sigma_p\) na borda \(r = R\) do dielétrico. Obtenha os campos \(\vetor{E},\) \(\vetor{D}\), e \(\vetor{P}\) para todo \(r > 0\).
\end{exercício}
\begin{proof}[Resolução]

\end{proof}
