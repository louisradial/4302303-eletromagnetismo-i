\begin{exercício}{}{exercício3}
    Considere duas placas metálicas paralelas infinitas separadas de uma distância \(d\) carregadas com densidades superficiais de carga uniformes \(\sigma_0\) (em \(z = d\)) e \(-\sigma_0\) (em \(z = 0\)).
    \begin{enumerate}[label=(\alph*)]
        \item Calcule o campo elétrico e o potencial elétrico na região entre as placas. Em seguida, calcule a densidade de energia \(u_\mathrm{E}\).
        \item Considere agora que um dielétrico linear de permissividade \(\epsilon\) preenche completamente a região entre as placas (as densidades de carga livre nas placas permanecem as mesmas). Obtenha nesse caso os campos \(\vetor{E}, \vetor{D}\) e \(\vetor{P}\) na região entre as placas, e em seguida calcule a nova densidade de energia, dada por
            \begin{equation*}
                u_\mathrm{E} = \frac12 \vetor{D}\cdot \vetor{E},
            \end{equation*}
            e compare-a com a obtida no item (a).
        \item Refaça o problema considerando agora que o dielétrico preenche apenas \emph{metade} do espaço entre as placas, na região \(0 < z < \frac{d}2\).
    \end{enumerate}
\end{exercício}
\begin{proof}[Resolução]

\end{proof}
