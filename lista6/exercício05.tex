\begin{exercício}{}{exercício5}
    Considere duas cascas esféricas metálicas concêntricas de raios \(a\) e \(b\), com \(a < b\). A casca de raio \(b\) está aterrada enquanto que a casca de raio \(a\) encontra-se mantida a um potencial \(\phi_0 > 0\). Na região entre as cascas há um dielétrico linear de constante dielétrica \(k\) preenchando apenas metade da região, como mostra a figura abaixo, onde \(\epsilon = k \epsilon_0\).

    \begin{enumerate}[label=(\alph*)]
        \item Obtenha o potencial elétrico e o campo elétrico nas regiões \(r < a\) e \(r > b\).
        \item Encontre os campos \(\vetor{E}\), \(\vetor{D}\), e \(\vetor{P}\) na região entre as cascas \(a < r < b\).
        \item Calcule todas as densidades de carga livre e de polarização.
        \item Com a resposta dos item anteriores, você é capaz de resolver \enquote{automaticamente} o problema ilustrado abaixo, onde o ângulo associado ao preenchimento do dielétrico não é \(\pi\)?
        \item Com a resposta dos itens anteriores, você é capaz de resolver \enquote{automaticamente} o problema ilustrado abaixo, onde há apenas uma casca esférica de raio \(a\) (mantida a um potencial \(\phi_0\)) e um dielétrico preenchendo \emph{todo} o hemisfério sul para \(r > a\)?
    \end{enumerate}
\end{exercício}
\begin{proof}[Resolução]

\end{proof}
