\begin{exercício}{}{exercício4}
    Considere duas placas metálicas quadradas de lado \(L\) e espessuras desprezíveis. As placas estão paralelas e separadas por uma distância \(d\). Assume que os planos das placas são paralelos ao plano \(xy\), que a placa inferior (superior) está em \(z = 0\) (\(z = d\)) e que estão orientadas como mostra a figura abaixo. A placa superior está mantida a um potencial \(\phi_0>0\), enquanto que a placa inferior está aterrada. Assumamos que \(d \ll L\), de forma que podemos desprezar quaisquer efeitos de borda.

    Adicionalmente, um dielétrico de constante dielétrica \(k\) preenche \emph{metade} da região entre as placas, e duas configurações serão consideradas, como mostram as figuras abaixo, onde \(\epsilon = k \epsilon_0\).

    Calcule a energia eletrostática armazenada e m toda a região entre as placas em cada configuração. Dado que \(k > 1\), em qual das duas configurações a energia é maior?
\end{exercício}
\begin{proof}[Resolução]

\end{proof}
