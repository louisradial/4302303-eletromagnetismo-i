\begin{exercício}{Momento linear associado ao campo eletromagnético}{exercício7}
    O momento linear associado ao campo eletromagnético pode ser definido como
    \begin{equation*}
        \vetor{p}_{\mathrm{EM}} = \epsilon_0 \mu_0 \int \dln3\x \vetor{S},
    \end{equation*}
    em que \(\vetor{S} = \frac{1}{\mu_0} \vetor{E}\times\vetor{B}\) é o vetor de Poynting. Isto posto, uma esfera de raio \(R\) é constituída de um material que contém uma polarização uniforme \(\vetor{P}\) e uma magnetização uniforme \(\vetor{M}\).
    \begin{enumerate}[label=(\alph*)]
        \item Calcule o momento linear armazenado no campo eletromagnético dentro da esfera.
        \item Calcule o momento linear armazenado no campo eletromagnético em todo o espaço.
    \end{enumerate}
\end{exercício}
\begin{proof}[Resolução]

\end{proof}
