\begin{exercício}{}{exercício1}
    Duas placas metálicas paralelas estão distantes de uma quantidade \(d\) em uma configuração onde, entre elas, há um material de condutividade \(\sigma\). Uma das placas é mantida a um potencial \(\phi_0 > 0\), enquanto que a outra encontra-se aterrada. Adicionalmente, uma pequena esfera metálica de raio \(a\ll d\) é cortada ao meio e acoplada -- com contato elétrico -- na placa aterrada, como mostra a figura abaixo.
    \begin{center}
        \begin{tikzpicture}
            \fill[Teal!35] (0.05,0) rectangle (4.95,3);
            \draw[very thick] (0,0) -- (5,0);
            \draw[very thick] (0,3) -- (5,3);
            \filldraw[fill=Sky!5, very thick] (3.5,0) arc[start angle=0, end angle=180, radius=1];
            \draw[thick] (2.5,0) -- ++(45:1);
            \node at (.5,2) {\( \sigma \)};
            \node[right] at (5,3) {\( \varphi = \varphi_0 \)};
            \node[right] at (5,0) {\( \varphi = 0 \)};
            \node[above left] at (3,0.3) {\( a \)};
            \draw[<->] (-0.3,0) -- (-0.3,3) node[midway,left] {\( d \)};
        \end{tikzpicture}
    \end{center}
    Considerando que a condutividade do material é pequena o suficiente para que a Lei de Ohm seja válida e desprezando-se quaisquer efeitos de borda nas laterais, obtenha a corrente elétrica que atravessa (apenas) a superfície da pequena semi-esfera.
\end{exercício}
\begin{proof}[Resolução]

\end{proof}
