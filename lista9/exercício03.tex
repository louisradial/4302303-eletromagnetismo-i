\begin{exercício}{Corrente induzida em uma espira coaxial a um solenoide longo}{exercício3}
    Um longo solenoide de raio \(a\) conduz uma corrente elétrica que varia no tempo de tal forma que o campo magnético gerado por ela no interior do solenoide é do tipo \(\vetor{B} = B_0 \cos(\omega t) \vetor{e}_z\). Em sobreposição a esse arranjo, uma espira circular de raio \(\frac12a\) e resistência \(R\) é colocada no interior deste solenoide, de forma que seu eixo de simetria coincide com o eixo de simetria do solenoide. Calcule a indutância mútua entre os dois objetos e encontre a corrente induzida na espira \(I(t)\).
\end{exercício}
\begin{proof}[Resolução]

\end{proof}
