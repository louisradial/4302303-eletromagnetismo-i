\begin{exercício}{Corrente induzida em uma espira coaxial a um solenoide longo}{exercício3}
    Um longo solenoide de raio \(a\) conduz uma corrente elétrica que varia no tempo de tal forma que o campo magnético gerado por ela no interior do solenoide é do tipo \(\vetor{B} = B_0 \cos(\omega t) \vetor{e}_z\). Em sobreposição a esse arranjo, uma espira circular de raio \(\frac12a\) e resistência \(R\) é colocada no interior deste solenoide, de forma que seu eixo de simetria coincide com o eixo de simetria do solenoide. Calcule a indutância mútua entre os dois objetos e encontre a corrente induzida na espira \(I(t)\).
\end{exercício}
\begin{proof}[Resolução]
    Consideremos uma corrente \(I'\) estacionária no solenoide, de modo que o campo magnético no interior do solenoide seja dado por
    \begin{equation*}
        \vetor{B}(\vetor{\x}) = \mu_0 n I' \vetor{e}_z,
    \end{equation*}
    onde \(n\) é o número de espiras por unidade de comprimento. Dessa forma, o fluxo magnético pela espira circular é dado por
    \begin{equation*}
        \Phi_B = \frac14\mu_0 \pi a^2 n I',
    \end{equation*}
    e concluímos que a indutância mútua entre o solenoide e a espira é \(M = \frac14\mu_0 \pi a^2 n\).

    Se o campo no interior do solenoide é \(\vetor{B} = B_0 \cos(\omega t)\), a corrente no solenoide é dada por
    \begin{equation*}
        I(t) = \frac{B_0}{\mu_0 n}\cos(\omega t).
    \end{equation*}
    Assim, o fluxo na espira é \(\Phi_B = M I(t)\), portanto
    \begin{equation*}
        I_\mathrm{induzida}(t) = \frac{\omega\pi a^2 B_0 \sin(\omega t)}{R}
    \end{equation*}
    é a corrente induzida na espira.
\end{proof}
