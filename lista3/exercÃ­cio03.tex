\begin{exercício}{Problema clássico do método das imagens}{exercício3}
    Uma carga pontual \(q\) é posicionada a uma distância \(a\) acima de um plano condutor aterrado. Encontre o potencial em todo o espaço. Utilize sua resposta para encontrar a função de Green \(G_D(\vetor{\x}, \vetor{\x'})\) específica para essa superfície de contorno.
\end{exercício}
\begin{proof}[Resolução]
    Sejam \(\vetor{u}\) um versor ortogonal ao plano fixo,
    \begin{equation*}
        \Omega_+ = \setc{\vetor{\x} \in \mathbb{R}^3}{\inner{\vetor{u}}{\vetor{\x}} > 0},\quad
        \Omega_- = \setc{\vetor{\x} \in \mathbb{R}^3}{\inner{\vetor{u}}{\vetor{\x}} < 0},\quad\text{e}\quad
        \Sigma = \setc{\vetor{\x} \in \mathbb{R}^3}{\inner{\vetor{u}}{\vetor{\x}} = 0}.
    \end{equation*}
    Notemos temos o problema de Dirichlet trivial em \(\Omega_-\), portanto \(\phi(\vetor{\x}) = 0\) para todo \(\vetor{\x} \in \Omega_- \cup \Sigma\). Em \(\Omega_+\), temos o problema de Dirichlet \(\phi(\x) = 0\) para todo \(\vetor{\x} \in \partial \Omega_+\) com \(\rho(\vetor{\x}) = q \delta(\vetor{\x} - a\vetor{u})\) para todo \(\vetor{\x} \in \Omega_+\). Sendo \(G_D(\vetor{\x}, \vetor{\x'})\) a função de Green para essa superfície de contorno, temos
    \begin{align*}
        \phi(\vetor{\x}) &= \frac{1}{4\pi \epsilon_0} \int_{\Omega_+} \dln3{\x'} \rho(\vetor{\x'})G_D(\vetor{\x}, \vetor{\x'}) - \frac1{4\pi} \oint_{\partial \Omega_+} \dln2{\x'} \phi(\vetor{\x'})\diffp{G_D(\vetor{\x}, \vetor{\x'})}{n'}\\
                         &=  \frac{q}{4\pi \epsilon_0} G_D(\vetor{\x}, a \vetor{u})
    \end{align*}
    como a solução formal para a equação de Poisson em \(\Omega_+\) com estas condições de contorno.

    Consideremos agora o problema auxiliar em que o plano condutor está ausente, mas com a presença de uma carga \(-q\) em \(a \vetor{u} - 2\inner{\vetor{u}}{a\vetor{u}}\vetor{u} = -a \vetor{u}\). O potencial \(\tilde{\phi}\) neste problema auxiliar é dado por
    \begin{equation*}
        \tilde{\phi}(\vetor{\x}) = \frac{q}{4\pi \epsilon_0} \left(\frac{1}{\norm{\vetor{\x} - a \vetor{u}}} - \frac{1}{\norm{\vetor{\x} + a\vetor{u}}}\right),
    \end{equation*}
    logo temos que \(\tilde{\phi}(\vetor{\x}) = 0\) para todo \(\vetor{\x} \in \Sigma\). Para todo \(\vetor{\x} \in \Omega_+\), há \(\tilde{\rho}(\vetor{\x}) = q \delta(\vetor{\x} - a\vetor{u})\).

    Pela unicidade de soluções para o problema de Dirichlet, o potencial
    \begin{equation*}
        \phi(\vetor{\x}) = \begin{cases}
            0,&\text{se }\vetor{\x} \in \Omega_- \cup \Sigma\\
            \dfrac{q}{4\pi \epsilon_0} \left(\dfrac{1}{\norm{\vetor{\x} - a \vetor{u}}} - \dfrac{1}{\norm{\vetor{\x} + a\vetor{u}}}\right),&\text{se }\vetor{\x} \in \Omega_+
        \end{cases}
    \end{equation*}
    é a solução para a equação de Poisson considerada. Por conseguinte, segue que
    \begin{equation*}
        G_D(\vetor{\x}, \vetor{\x'}) = \frac{1}{\norm{\D}} - \frac{1}{\norm{\vetor{\x} - \vetor{\x'} + 2\inner{\vetor{u}}{\vetor{\x'}}\vetor{u}}}
    \end{equation*}
    é a função de Green para esta superfície de contorno.
\end{proof}
