\begin{exercício}{Plano isolante infinito}{exercício4}
    Um plano isolante infinito é posicionado sobre o plano \(xy\). O material que constitui o plano é feito de forma que o potencial sobre seus pontos é dado pela expressão
    \begin{equation*}
        \phi_P(x,y) = \phi_0 \exp\left(- \frac{x^2 + y^2}{2 \alpha^2}\right),
    \end{equation*}
    em que \(\phi_0\) e \(\alpha\) são constantes reais e positivas. Encontre o potencial em todos os pontos do eixo \(z\) para \(z > 0\).
\end{exercício}
\begin{proof}[Resolução]

\end{proof}
