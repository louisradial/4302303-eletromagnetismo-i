\begin{exercício}{Casca esférica condutora e carga interna}{exercício8}
    Refaça os \cref{ex:exercício5,ex:exercício6,ex:exercício7} considerando uma casca esférica condutora no lugar da esfera e que a carga pontual é colocada em seu interior a uma distância \(d < R\) do centro. Cheque a validade das respostas no limite \(d \to 0\).
\end{exercício}
\begin{proof}[Resolução para a esfera aterrada]
    Mantenhamos a notação do \cref{ex:exercício5}, com a diferença de que a distribuição de cargas é dada por
    \begin{equation*}
        \rho(\vetor{\x}) = \begin{cases}
            q \delta(\vetor{\x} - \vetor{\x}_q), &\text{se }\vetor{\x} \in \Sigma\\
            0, &\text{se }\vetor{\x} \in \Omega,
        \end{cases}
    \end{equation*}
    com \(\vetor{\x}_q \neq \vetor{0}\). Nesta situação o potencial satisfaz a equação de Laplace em \(\Omega\) com \(\phi(\vetor{\x})= 0\) para todo \(\vetor{\x} \in \partial \Omega\), logo \(\phi(\vetor{\x}) = 0\) para todo \(\vetor{\x} \in \Omega\).

    Consideremos o problema auxiliar em que há duas cargas \(q\) e \(\tilde{q}\) nas posições \(\vetor{\x}_q\) e \(\vetor{\x}_{\tilde{q}}\), com \(\vetor{\x}_{\tilde{q}} \in \Omega\) de modo que a distribuição de cargas em \(\Sigma\) seja igual ao problema original. Desejamos determinar \(\tilde{q}\) e \(\vetor{\x}_{\tilde{q}}\) tais que \(\phi(\vetor{\x}) = 0\) para todo \(\vetor{\x} \in \partial \Sigma\). Sejam \(r = \norm{\vetor{\x}}\), \(\tilde{d} = \norm{\vetor{\x}_{\tilde{q}}}\), \(\gamma\) o ângulo planar entre \(\vetor{\x}\) e \(\vetor{\x}_q\), e \(\tilde{\gamma}\) o ângulo planar entre \(\vetor{\x}\) e \(\vetor{\x}_{\tilde{q}}\), então
    \begin{align*}
        \phi(\vetor{\x}) &= \frac{1}{4\pi \epsilon_0} \left(\frac{q}{\sqrt{r^2 + d^2 - 2rd \cos \gamma}} + \frac{\tilde{q}}{\sqrt{r^2 + \tilde{d}^2 - 2r\tilde{d} \cos\tilde{\gamma}}}\right)\\
                         &= \frac{1}{4\pi \epsilon_0} \left[\frac{q}{d\sqrt{1 + \left(\frac{r}{d}\right)^2 - 2 \left(\frac{r}{d}\right)\cos \gamma}} + \frac{\tilde{q}}{r\sqrt{1 + \left(\frac{\tilde{d}}{r}\right)^2 - 2 \left(\frac{\tilde{d}}{r}\right)\cos\tilde{\gamma}}}\right]
    \end{align*}
    é o potencial para o problema auxiliar. Por inspeção, vemos que
    \begin{equation*}
        \tilde{q} = - \frac{R}{d}q,
        \quad
        \tilde{d} = \frac{R^2}{d},
        \quad\text{e}\quad
        \tilde{\gamma} = \gamma
    \end{equation*}
    satisfazem \(\phi(\vetor{\x}) = 0\) para todo \(\vetor{\x} \in \partial \Sigma\), como desejado. Isto é, \(\vetor{\x}_{\tilde{q}} = \frac{R^2}{\norm{\vetor{\x}_q}^2}\vetor{\x}_q\) e o potencial é dado por
    \begin{equation*}
        \phi(\vetor{\x}) = \frac{q}{4\pi \epsilon_0} \left(\frac{1}{\norm{\vetor{\x} - \vetor{\x}_q}} - \frac{R}{\norm{\vetor{\x}_q}\norm*{\vetor{\x} - \frac{R^2}{\norm{\vetor{\x}_q}^2}\vetor{\x}_q}}\right)
    \end{equation*}
    para todo \(\vetor{\x} \in \mathbb{R}^3\) no problema auxiliar. Em particular esta expressão é válida para todo \(\vetor{\x} \in \Sigma\) e satisfaz as condições de contorno do problema original, portanto podemos concluir que
    \begin{equation*}
        \phi(\vetor{\x}) = \begin{cases}
            0, &\text{se } \vetor{\x}\in \Omega \cup \partial \Sigma\\
            \frac{q}{4\pi \epsilon_0} \left(\frac{1}{\norm{\vetor{\x} - \vetor{\x}_q}} - \frac{R}{\norm{\vetor{\x}_q}\norm*{\vetor{\x} - \frac{R^2}{\norm{\vetor{\x}_q}^2}\vetor{\x}_q}}\right),&\text{se }\vetor{\x} \in \Sigma
        \end{cases}
    \end{equation*}
    é o potencial do problema original pela unicidade de soluções para a equação de Poisson.

    Sendo \(G_D(\vetor{\x}, \vetor{\x'})\) a função de Green para essa superfície de contorno, temos
    \begin{align*}
        \phi(\vetor{\x}) &= \frac{1}{4\pi \epsilon_0}\int_{\Sigma} \dln3{\x'} \rho(\vetor{\x'})G_D(\vetor{\x}, \vetor{\x'}) - \frac1{4\pi}\oint_{\partial \Sigma}\dln2{\x'} \phi(\vetor{\x'})\diffp{G_D(\vetor{\x}, \vetor{\x'})}{n'}\\
                         &= \frac{q}{4\pi \epsilon_0} \int_{\Sigma} \dln3{\x'} \delta(\vetor{\x'}-\vetor{\x}_q)G_D(\vetor{\x}, \vetor{\x'})\\
                         &= \frac{q}{4\pi \epsilon_0} G_D(\vetor{\x}, \vetor{\x}_q).
    \end{align*}
    Assim, concluímos que
    \begin{equation*}
        G_D(\vetor{\x}, \vetor{\x'}) = \frac{1}{\norm{\vetor{\x} - \vetor{\x'}}} - \frac{R}{\norm{\vetor{\x'}}\norm*{\vetor{\x} - \frac{R^2}{\norm{\vetor{\x'}}^2}\vetor{\x'}}},
        \quad\text{com}\quad
        0 < \norm{\vetor{\x'}} < R,
    \end{equation*}
    é a função de Green para essa superfície de contorno.

    Para determinar a densidade de carga induzida na superfície do condutor, orientemos o eixo \(z\) de acordo com a posição da carga \(q\). Temos
    \begin{align*}
        \diffp{\phi}{r}[r=R^-] &= \frac{q}{4\pi \epsilon_0} \diffp*{\left(\frac{1}{\sqrt{r^2 + d^2 - 2rd \cos\theta}} - \frac{R}{d\sqrt{r^2 + \frac{R^4}{d^2} - 2 \frac{rR^2}{d}\cos\theta}}\right)}{r}[r=R]\\
                               &= -\frac{q}{4\pi \epsilon_0} \left[\frac{r - d\cos\theta}{(r^2 + d^2 - 2rd\cos\theta)^{\frac32}} - \frac{R\left(r - \frac{R^2}{d}\cos\theta\right)}{d\left(r^2 + \frac{R^4}{d^2} - 2\frac{rR^2}{d}\cos\theta\right)^{\frac32}}\right]_{r=R}\\
                               &= -\frac{q}{4\pi \epsilon_0} \left[\frac{R - d\cos\theta}{(R^2 + d^2 - 2Rd\cos\theta)^{\frac32}} - \frac{R^2\left(d - R\cos\theta\right)}{d^2\left(\frac{R^2}{d^2}\right)^{\frac32}\left(d^2 + R^2 - 2Rd\cos\theta\right)^{\frac32}}\right]\\
                               &= -\frac{q}{4\pi \epsilon_0 (R^2 + d^2 - 2Rd \cos\theta)^{\frac32}}\left[R - d \cos\theta - \frac{R^2d^3(d - R\cos\theta)}{d^2R^3}\right]\\
                               &= \frac{(d^2 - R^2)q}{4\pi \epsilon_0 R(R^2 + d^2 - 2Rd \cos\theta)^{\frac32}},
    \end{align*}
    portanto a distribuição de cargas na superfície do condutor é
    \begin{equation*}
        \sigma = -\epsilon_0\left(\diffp{\phi}{r}[r=R^+] - \diffp{\phi}{r}[r=R^{-}]\right)=\frac{(d^2 - R^2)q}{4\pi R(R^2 + d^2 - 2Rd \cos\theta)^{\frac32}}.
    \end{equation*}
    Assim, a carga total \(Q\) induzida é
    \begin{align*}
        Q = \int_0^\pi R\dli{\theta} \int_0^{2\pi} R\sin\theta \dli{\varphi} \sigma(\theta, \varphi)
        &= \frac{(d^2 - R^2)Rq}{2}\int_0^\pi \dli{\theta} \frac{\sin\theta}{(R^2 + d^2 - 2Rd\cos\theta)^{\frac32}}\\
        &= \frac{(R^2 - d^2)q}{4d}\int_{(R+d)^2}^{(R-d)^2} \dli{\xi} \xi^{-\frac32}\\
        &= -\frac{(R^2 - d^2)q}{2d}\left(\frac{1}{R - d} - \frac{1}{R+d}\right)\\
        &= -q.
    \end{align*}
    Ainda, a força elétrica devido à distribuição de cargas induzidas na carga \(q\) é dada por
    \begin{equation*}
        \vetor{F} = \frac{-Rq^2\left(\vetor{\x}_q - \frac{R^2}{\norm{\vetor{\x}_q}^2}\vetor{\x}_q\right)}{4\pi \epsilon_0\norm{\vetor{\x}_q}\norm*{\vetor{\x}_q - \frac{R^2}{\norm{\vetor{\x}_q}^2}\vetor{\x}_q}^3} = - \frac{Rq^2\left(1 - \frac{R^2}{\norm{\vetor{\x}_q}}\right)}{4\pi \epsilon_0\abs*{1 - \frac{R^2}{\norm{\vetor{\x}_q}^2}}^3\norm{\vetor{\x}_q}^4}\vetor{\x}_q = \frac{Rq^2}{4\pi \epsilon_0 \left(R^2 - \norm{\vetor{\x}_q}^2\right)^2}\vetor{\x}_q.
    \end{equation*}

    No limite em que \(\norm{\vetor{\x}_q} \to 0,\) temos \(\vetor{F} \to \vetor{0}\), \(\sigma \to -\frac{q}{4\pi R^2}\) e
    \begin{equation*}
        \phi(\vetor{\x}) \to \begin{cases}
            0, &\text{se }\vetor{\x} \in \Omega \cup \partial \Sigma\\
            \dfrac{q}{4\pi \epsilon_0} \left(\dfrac{1}{\norm{\vetor{\x}}} - \dfrac{1}{R}\right), &\text{se }\vetor{\x} \in \Sigma,
        \end{cases}
    \end{equation*}
    pois temos
    \begin{align*}
        \lim_{\norm{\vetor{\x}_q}\to 0} \frac{\frac{R}{\norm{\vetor{\x}_q}}}{\norm*{\vetor{\x} - \frac{R^2}{\norm{\vetor{\x}_q}^2}\vetor{\x}_q}}
        = \lim_{d \to 0^+} \frac{\frac{R}{d}}{\sqrt{r^2 + \frac{R^4}{d^2} - 2r\frac{R^2}{d}\cos \gamma}}
        = \lim_{d \to 0^+} \frac{\frac{R}{d}}{\frac{R}{d}\sqrt{\frac{d^2r^2}{R^2} + R^2 - 2rd\cos \gamma}} = \frac{1}{R}.
    \end{align*}
    Notemos que estes resultados são justamente aqueles obtidos para o caso em que a carga \(q\) está no centro da esfera. De fato, o limite para o potencial satisfaz a equação de Poisson em \(\Sigma\) com condição de contorno de que o potencial se anula em \(\partial \Sigma\) e satisfaz a equação de Laplace com a condição de que o potencial se anula em \(\partial \Omega\), portanto o limite obtido é o potencial para este caso particular e os demais resultados são facilmente constatados.

    Como a função de Green obtida para o caso em que a carga \(q\) não está no centro da esfera é a mesma que a obtida no \cref{ex:exercício5}, temos
    \begin{equation*}
        G_D(\vetor{\x}, \vetor{\x'}) = \begin{cases}
            \dfrac{1}{\norm{\vetor{\x} - \vetor{\x'}}} - \dfrac{R}{\norm{\vetor{\x'}}\norm*{\vetor{\x} - \frac{R^2}{\norm{\vetor{\x'}}^2}\vetor{\x'}}},&\text{se }\vetor{\x'}\in \Sigma \cup \Omega \setminus \set{\vetor{0}}\\
            \dfrac{1}{\norm{\vetor{\x}}} - \dfrac{1}{R},&\text{se }\vetor{\x'} = \vetor{0},
        \end{cases}
    \end{equation*}
    como a função de Green para uma superfície esférica de raio \(R\).
\end{proof}

\begin{proof}[Resolução para a esfera mantida a potencial constante]
    No \cref{ex:exercício6}, encontramos uma função \(\tilde{\phi}(\vetor{\x})\) que satisfaz a equação de Laplace com condições de contorno de que \(\tilde{\phi}(\vetor{\x}) = \phi_0\) para todo \(\vetor{\x} \in \partial \Sigma\) e que \(\tilde{\phi}(\vetor{\x}) \to 0\) conforme \(\norm{\vetor{\x}} \to \infty\). Desse modo, a soma desta função com o potencial encontrado para o caso de aterramento satisfaz a equação de Poisson em \(\Sigma\) e de Laplace em \(\Omega\) com as devidas condições de contorno. Pela unicidade de soluções, concluímos que
    \begin{equation*}
        \phi(\vetor{\x}) = \begin{cases}
            \dfrac{\phi_0R}{\norm{\vetor{\x}}}, &\text{se } \vetor{\x}\in \Omega \cup \partial \Sigma\\
            \dfrac{q}{4\pi \epsilon_0} \left(\dfrac{1}{\norm{\vetor{\x} - \vetor{\x}_q}} - \dfrac{R}{\norm{\vetor{\x}_q}\norm*{\vetor{\x} - \frac{R^2}{\norm{\vetor{\x}_q}^2}\vetor{\x}_q}}\right) + \phi_0,&\text{se }\vetor{\x} \in \Sigma\text{ e }\vetor{\x}_q \neq \vetor{0}\\
            \dfrac{q}{4\pi \epsilon_0} \left(\dfrac{1}{\norm{\vetor{\x}}} - \dfrac{1}{R}\right) + \phi_0,&\text{se }\vetor{\x} \in \Sigma\text{ e }\vetor{\x}_q = \vetor{0}
        \end{cases}
    \end{equation*}
    é o potencial em todo o espaço para o problema em que a esfera é mantida a um potencial \(\phi_0\). Assim, a densidade de carga superficial na esfera é dada por
    \begin{equation*}
        \sigma = -\epsilon_0\left(\diffp{\phi}{r}[r=R^+] - \diffp{\phi}{r}[r=R^{-}]\right)= \begin{cases}
            \dfrac{\epsilon_0\phi_0}{R} + \dfrac{(d^2 - R^2)q}{4\pi R(R^2 + d^2 - 2Rd \cos\theta)^{\frac32}} ,&\text{se }\vetor{\x}_q \neq \vetor{0}\\
            \dfrac{\epsilon_0 \phi_0}{R} - \dfrac{q}{4\pi R^2},&\text{se }\vetor{\x}_q = \vetor{0},
        \end{cases}
    \end{equation*}
    com carga total \(Q = 4\pi \epsilon_0 \phi_0 R - q\). Como o \(\tilde{\phi}(\vetor{\x})\) em \(\Sigma\) é constante, segue que
    \begin{equation*}
        \vetor{F} = \begin{cases}
            \dfrac{q^2R}{4\pi \epsilon_0\left(R^2 - \norm{\vetor{\x}_q}^2\right)^2}\vetor{\x}_q, &\text{se }\vetor{\x}_q\neq \vetor{0}\\
            \vetor{0},&\text{se }\vetor{\x}_q = \vetor{0}
        \end{cases}
    \end{equation*}
    é a força elétrica sobre a carga pontual, como no caso da esfera aterrada.
\end{proof}

\begin{proof}[Resolução para a esfera condutora isolada e carregada]
    Como a superfície do condutor é uma equipotencial, existe um \(\phi_0 \in \mathbb{R}\) tal que \(\phi(\vetor{\x}) = \phi_0\) para todo \(\vetor{\x} \in \partial \Sigma\). Esta situação é, portanto, equivalente ao caso anterior, em que a esfera é mantida a um potencial \(\phi_0\), e este potencial é unicamente determinado pela carga total \(Q\), com
    \begin{equation*}
        \phi_0 = \frac{q+Q}{4\pi \epsilon_0 R}.
    \end{equation*}
    Substituindo este resultado nas expressões do caso anterior, temos
    \begin{equation*}
        \phi(\vetor{\x}) = \begin{cases}
            \dfrac{q + Q}{4\pi \epsilon_0\norm{\vetor{\x}}}, &\text{se } \vetor{\x}\in \Omega \cup \partial \Sigma\\
            \dfrac{1}{4\pi \epsilon_0} \left(\dfrac{q}{\norm{\vetor{\x} - \vetor{\x}_q}} - \dfrac{qR}{\norm{\vetor{\x}_q}\norm*{\vetor{\x} - \frac{R^2}{\norm{\vetor{\x}_q}^2}\vetor{\x}_q}}+ \dfrac{q + Q}{R}\right) ,&\text{se }\vetor{\x} \in \Sigma\text{ e }\vetor{\x}_q \neq \vetor{0}\\
            \dfrac{1}{4\pi \epsilon_0} \left(\dfrac{q}{\norm{\vetor{\x}}} + \dfrac{Q}{R}\right),&\text{se }\vetor{\x} \in \Sigma\text{ e }\vetor{\x}_q = \vetor{0}
        \end{cases}
    \end{equation*}
    como a expressão para o potencial em todo o espaço. Segue também que
    \begin{equation*}
        \sigma = -\epsilon_0\left(\diffp{\phi}{r}[r=R^+] - \diffp{\phi}{r}[r=R^{-}]\right)= \begin{cases}
            \dfrac{q + Q}{4\pi R^2} + \dfrac{(d^2 - R^2)q}{4\pi R(R^2 + d^2 - 2Rd \cos\theta)^{\frac32}} ,&\text{se }\vetor{\x}_q \neq \vetor{0}\\
            \dfrac{Q}{4\pi R^2},&\text{se }\vetor{\x}_q = \vetor{0},
        \end{cases}
    \end{equation*}
    é a densidade de carga superficial na esfera, e que
    \begin{equation*}
        \vetor{F} = \begin{cases}
            \dfrac{q^2R}{4\pi \epsilon_0\left(R^2 - \norm{\vetor{\x}_q}^2\right)^2}\vetor{\x}_q, &\text{se }\vetor{\x}_q\neq \vetor{0}\\
            \vetor{0},&\text{se }\vetor{\x}_q = \vetor{0}
        \end{cases}
    \end{equation*}
    é a força elétrica sobre a carga pontual, como no caso da esfera aterrada.
\end{proof}
