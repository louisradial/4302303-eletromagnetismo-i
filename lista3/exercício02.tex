\begin{exercício}{Carga pontual entre placas condutoras}{exercício2}
    Considere dois condutores aterrados com bordas paralelas separados por uma distância \(d\), como mostra a figura abaixo. Os condutores são infinitos nas extensões \(x\) e \(y\). Ao longo da direção \(z\), o condutor 1 cobre todo o conjunto \(- \infty < z \leq 0\), enquanto que o condutor 2 cobre todo o conjunto \(d \leq z < \infty\). Uma carga pontual \(q\) é colocada no espaço entre eles, a uma distância \(a \in (0, d)\) da borda do condutor 1. O objetivo aqui é encontrar as cargas \emph{totais} induzidas nas superfícies dos condutores. Chamaremos a situação descrita de \emph{problema original}.

    \begin{center}
        \begin{tikzpicture}
            \fill[top color=Overlay2, bottom color=Base] (-7,0.5) rectangle (-1,2);
            \draw (-7,0.5) -- (-1,0.5) -- (-1,2) -- (-7,2) -- cycle;
            \node at (-4,1.25) {condutor 2};
            \node[anchor=north east] at (-1.2,1.9) {$\phi = 0$};

            \fill[top color=Base, bottom color=Overlay2] (-7,-2.75) rectangle (-1,-1.25);
            \draw (-7,-2.75) -- (-1,-2.75) -- (-1,-1.25) -- (-7,-1.25) -- cycle;
            \node at (-4,-2) {condutor 1};
            \node[anchor=north east] at (-1.2,-1.35) {$\phi = 0$};

            \fill (-4,-0.5) circle (2pt);
            \node at (-3.7,-0.5) {$q$};

            \draw[->] (-7.5,-2) -- (-7.5,1.4) node[above] {$z$};
            \draw[-] (-7.4,0.5) -- (-7.6,0.5) node[anchor=above, left] {$d$};
            \draw[-] (-7.4,-1.25) -- (-7.6,-1.25) node[anchor=above, left] {$0$};
            \draw[dashed] (-4,-0.5) -- (-7.5,-0.5) node[anchor=above, left] {\(a\)};

            \node at (-4,-3.25) {Problema original};

            \fill[top color=Lavender!50, bottom color=Base] (2,0.5) rectangle (8,2);
            \draw (2,0.5) -- (8,0.5) -- (8,2) -- (2,2) -- cycle;
            \node at (5,1.25) {condutor 2};
            \node[anchor=north east] at (7.8,1.9) {$\phi = \phi_0$};

            \fill[top color=Base, bottom color=Overlay2] (2,-2.75) rectangle (8,-1.25);
            \draw (2,-2.75) -- (8,-2.75) -- (8,-1.25) -- (2,-1.25) -- cycle;
            \node at (5,-2) {condutor 1};
            \node[anchor=north east] at (7.8,-1.35) {$\phi = 0$};

            \draw[->] (1.5,-2) -- (1.5,1.4) node[above] {$z$};
            \draw[-] (1.6,0.5) -- (1.4,0.5) node[anchor=above,left] {$d$};
            \draw[-] (1.6,-1.25) -- (1.4,-1.25) node[anchor=above,left] {$0$};

            \node at (5,-3.25) {Problema auxiliar};
        \end{tikzpicture}
    \end{center}
    Este é um problema complicado, pois não sabemos exatamente de que forma essas cargas se distribuem. Em outras palavras, não sabemos de antemão escrever as funções \(\sigma_1(x,y)\) e \(\sigma_2(x,y)\), e as cargas totais são as integrais dessas funções sobre as respectivas superfícies. Vamos então explorar a estratégia de considerar um \emph{problema auxiliar}, em que não há cargas pontuais na região entre os condutores e o condutor 2 está mantido a um potencial \(\phi_0 \neq 0\), como mostra a figura acima. Nesse caso, encontre o potencial em todo o espaço e as densidades de carga \(\sigma_1(x,y)\) e \(\sigma_2(x,y)\) nas superfícies dos condutores 1 e 2, respectivamente. Em seguida, utilize o \cref{ex:exercício1} para resolver o problema original.
\end{exercício}
\begin{proof}[Resolução]

\end{proof}
