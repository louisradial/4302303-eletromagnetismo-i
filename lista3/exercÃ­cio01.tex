\begin{exercício}{Teorema da reciprocidade de Green}{exercício1}
    Considere uma cavidade de forma arbitrária completamente cercada por um conjunto de estruturas condutoras, todas de formas também arbitrárias. Chame o volume da cavidade de \(\Omega\) e sua superfície (isto é, a fronteira entre a cavidade e a coleção de condutores) de \(\partial \Omega\). A figura abaixo mostra a ideia do arranjo, em que cada um dos condutores (cada um representado por uma cor) é mantido em seu próprio potencial, denotados por \(\phi_n\).

    \begin{center}
        \begin{tikzpicture}[scale=0.7]
        \foreach \i/\col in {1/Pink, 2/Mauve, 3/Sky, 4/Teal, 5/Red, 6/Yellow, 7/Lavender} {
            \fill[\col!50] (90+\i*360/7:4cm) -- (90+\i*360/7+360/7:4cm) --
                        (90+\i*360/7+360/7:2cm) -- (90+\i*360/7:2cm) -- cycle;
        }

        \fill[Overlay2!5] (0,0) circle (2cm);
        \draw[dashed] (0,0) circle (2cm);

        % Labels for the outer segments
        \node at (90+1*360/7+360/14:3cm) {$\phi_1$};
        \node at (90+2*360/7+360/14:3cm) {$\phi_2$};
        \node at (90+3*360/7+360/14:3cm) {$\phi_3$};
        \node at (90+4*360/7+360/14:3cm) {$\phi_4$};
        \node at (90+5*360/7+360/14:3cm) {$\phi_5$};
        \node at (90+6*360/7+360/14:3cm) {$\phi_6$};
        \node at (90+7*360/7+360/14:3cm) {$\phi_7$};

        \node at (0,0) {$\Omega$};
        \node at (1.5,0.6) {$\partial \Omega$};
        \end{tikzpicture}
    \end{center}
    Suponha que não haja contato elétrico entre os condutores. Considere agora \emph{duas} situações distintas:
    \begin{enumerate}[label=(\alph*)]
        \item os potenciais dos condutores são tais que o potencial é uma função \(\phi(\vetor{\x})\) em \(\partial \Omega\) e existe tanto uma densidade de carga \(\rho(\vetor{\x})\) no interior de \(\Omega\) quanto uma densidade superficial \(\sigma\) em \(\partial \Omega\).
        \item os potenciais dos condutores são tais que o potencial é uma função \(\tilde{\phi}(\vetor{\x})\) em \(\partial \Omega\) e existe tanto uma densidade de carga \(\tilde{\rho}(\vetor{\x})\) no interior de \(\Omega\) quanto uma densidade superficial de carga \(\tilde{\sigma}\) em \(\partial \Omega\).
    \end{enumerate}
    Em outras palavras, as situações (a) e (b) representam dois problemas de condições de contorno diferentes, com distribuições de cargas diferentes, sobre a \emph{mesma} estrutura (isto é, mesmo domínio; mesmo \(\Omega\) e \(\partial \Omega\)). Prove que
    \begin{equation*}
        \int_{\Omega}\dln3\x \phi(\vetor{\x}) \tilde{\rho}(\vetor{\x}) + \oint_{\partial \Omega}\dli{a} \phi(\vetor{\x})\tilde{\sigma}(\vetor{\x}) = \int_{\Omega}\dln3x\x \tilde{\phi}(\vetor{\x})\rho(\vetor{\x}) + \oint_{\partial \Omega} \dli{a} \tilde{\phi}(\vetor{\x})\sigma(\vetor{\x}),
    \end{equation*}
    igualdade por sua vez conhecida como teorema da reciprocidade de Green.
\end{exercício}
\begin{proof}[Resolução]

\end{proof}
