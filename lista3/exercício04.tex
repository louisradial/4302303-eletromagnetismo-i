\begin{exercício}{Plano isolante infinito}{exercício4}
    Um plano isolante infinito é posicionado sobre o plano \(xy\). O material que constitui o plano é feito de forma que o potencial sobre seus pontos é dado pela expressão
    \begin{equation*}
        \phi_P(x,y) = \phi_0 \exp\left(- \frac{x^2 + y^2}{2 \alpha^2}\right),
    \end{equation*}
    em que \(\phi_0\) e \(\alpha\) são constantes reais e positivas. Encontre o potencial em todos os pontos do eixo \(z\) para \(z > 0\).
\end{exercício}
\begin{proof}[Resolução]
    Sejam
    \begin{equation*}
        \Omega = \setc{\vetor{\x} \in \mathbb{R}^3}{\inner{\vetor{e}_z}{\vetor{\x}} > 0}
        \quad\text{e}\quad
        \Sigma = \setc{\vetor{\x} \in \mathbb{R}^3}{\inner{\vetor{u}}{\vetor{\x}} = 0}.
    \end{equation*}
    A solução para a equação de Laplace em \(\Omega\) com condição de contorno \(\phi(\vetor{\x}) = \phi_P(\inner{\vetor{e}_x}{\vetor{\x}}, \inner{\vetor{e}_y}{\vetor{\x}})\) para todo \(\vetor{\x} \in \Sigma\) é dada por
    \begin{align*}
        \phi(\vetor{\x}) &= -\frac1{4\pi} \oint_{\partial \Omega} \dli{a'} \phi(\vetor{\x'}) \diffp{G_D(\vetor{\x}, \vetor{\x'})}{n'}\\
                         &= -\frac1{4\pi} \int_{\Sigma} \dli{a'} \phi(\vetor{\x'}) \vetor{n'} \cdot \nabla'\left(\frac{1}{\norm{\D}} - \frac{1}{\norm{\vetor{\x} - \vetor{\x'} + 2\inner{\vetor{e}_z}{\vetor{\x'}}\vetor{e}_z}}\right)\\
                         &=-\frac1{4\pi} \int_{\Sigma} \dli{a'} \phi(\vetor{\x'}) \left[\frac{\inner{\vetor{n'}}{\D}}{\norm{\D}^3} -\vetor{n'}\cdot \nabla'\left(\frac{1}{\norm{\vetor{\x} - \vetor{\x'} + 2\inner{\vetor{e}_z}{\vetor{\x'}}\vetor{e}_z}}\right)\right],
    \end{align*}
    para todo \(\vetor{\x} \in \Omega\). Computemos a derivada direcional utilizando coordenadas cartesianas
    \begin{align*}
        \diffp*{\left(\frac{1}{\norm{\vetor{\x} - \vetor{\x'} + 2\inner{\vetor{e}_z}{\vetor{\x'}}\vetor{e}_z}}\right)}{z'}
        &= \diffp*{\left[(x - x')^2 + (y - y')^2 + (z + z')^2\right]^{-\frac12}}{z'}\\
        &= -\frac{z + z'}{\norm{\vetor{\x} - \vetor{\x'} + 2\inner{\vetor{e}_z}{\vetor{\x'}}\vetor{e}_z}^3},
    \end{align*}
    logo
    \begin{equation*}
        \vetor{n'} \cdot \nabla'\left(\frac{1}{\norm{\vetor{\x} - \vetor{\x'} + 2\inner{\vetor{e}_z}{\vetor{\x'}}\vetor{e}_z}}\right) = \frac{\inner{\vetor{e}_z}{\vetor{\x} - \vetor{\x'}+ 2\inner{\vetor{e_z}}{\vetor{\x'}}}}{\norm{\vetor{\x} - \vetor{\x'} + 2\inner{\vetor{e}_z}{\vetor{\x'}}\vetor{e}_z}^3}.
    \end{equation*}
    Assim,
    \begin{align*}
        \phi(x \vetor{e}_x + y \vetor{e}_y + z \vetor{e}_z)
        &= -\frac1{4\pi} \int_{\Sigma} \dli{a'} \phi(\vetor{\x'}) \left(\frac{\inner{\vetor{n'}}{\D}}{\norm{\D}^3} -\frac{\inner{\vetor{e_z}}{\vetor{\x} + \vetor{\x'}}}{\norm{\vetor{\x} - \vetor{\x'} + 2\inner{\vetor{e}_z}{\vetor{\x'}}\vetor{e}_z}^3}\right)\\
        &= \frac{\phi_0 z}{2\pi} \int_{\mathbb{R}} \dli{x'} \int_{\mathbb{R}} \dli{y'} \frac{\exp\left(-\frac{x'^2 + y'^2}{2 \alpha^2}\right)}{\left[(x - x')^2 + (y-y')^2 + z^2\right]^{\frac32}}
    \end{align*}
    é o potencial para todo \((x,y,z) \in \Omega\). Em particular,
    \begin{equation*}
        \phi(z \vetor{e}_z) = \phi_0 z \int_0^\infty \dli{r} \frac{r}{\left(r^2 + z^2\right)^{\frac32}} \exp\left(-\frac{r^2}{2 \alpha^2}\right)
    \end{equation*}
    é o potencial ao longo do eixo \(z\).

    Consideremos a mudança de variáveis \(2 \alpha^2\xi^2 = r^2 + z^2\), então
    \begin{align*}
        \phi(z \vetor{e}_z) &= \phi_0 \frac{z}{\alpha \sqrt{2}} \exp\left(\frac{z^2}{2 \alpha^2}\right) \int_{\frac{z}{\alpha \sqrt{2}}}^\infty \dli{\xi} \xi^{-2} \exp\left(-\xi^2\right)\\
                            &= \phi_0 \frac{z}{\alpha \sqrt{2}} \exp\left(\frac{z^2}{2 \alpha^2}\right) \left\{\left[-\xi^{-1}\exp\left(-\xi^2\right)\right]_{\frac{z}{\alpha \sqrt{2}}}^\infty - 2\int_{\frac{z}{\alpha \sqrt{2}}}^\infty \dli{\xi}\exp(-\xi^2)\right\}\\
                            &= \phi_0  \frac{z}{\alpha \sqrt{2}} \exp\left(\frac{z^2}{2 \alpha^2}\right)\left[\frac{\alpha \sqrt{2}}{z}\exp\left(-\frac{z^2}{2 \alpha^2}\right) - \sqrt{\pi}\operatorname{erfc}\left(\frac{z}{\alpha\sqrt{2}}\right)\right]\\
                            &= \phi_0 \left[1 - \sqrt{\frac{\pi z^2}{2 \alpha^2}} \exp\left(\frac{z^2}{2 \alpha^2}\right)\operatorname{erfc}\left(\frac{z}{\alpha\sqrt{2}}\right)\right],
    \end{align*}
    onde \(\operatorname{erfc}\) é a função de erro complementar \(\operatorname{erfc}(x) = \frac{2}{\sqrt{\pi}}\int_x^\infty \dli{t} \exp(-t^2)\).
\end{proof}
