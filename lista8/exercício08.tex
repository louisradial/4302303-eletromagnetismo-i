\begin{exercício}{Dipolo magnético no centro de uma esfera de material magnético linear}{exercício8}
    Um dipolo magnético ideal do tipo \(\vetor{m} = m\vetor{e}_z\) está localizado no centro de uma esfera maciça de raio \(R\) feita de um material magnético linear de permeabilidade \(\mu\). Mostre que o campo magnético na região \(r \in (0,R)\) é dado por
    \begin{equation*}
        \vetor{B}(\vetor{\x}) = \frac{\mu}{4\pi\norm{\vetor{\x}}^5}\left[3\inner{\vetor{m}}{\vetor{\x}}\vetor{\x} - \norm{\vetor{\x}}^2\vetor{m}\right] + \frac{\mu(\mu_0 - \mu)}{2\pi R^3(2\mu_0 + \mu)}\vetor{m}.
    \end{equation*}
    Qual é o campo fora da esfera?
\end{exercício}
\begin{proof}[Resolução]

\end{proof}
