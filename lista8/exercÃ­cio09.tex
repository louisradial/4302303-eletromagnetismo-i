\begin{exercício}{Ímã cilíndrico}{exercício9}
    Um cilindro maciço de raio \(R\) e comprimento \(L\) possui uma magnetização constante \(\vetor{M} = M \vetor{e}_z\), onde o eixo \(z\) e a origem coincidem, respectivamente, com o eixo de simetria e o centro do cilindro. Você pode pensar nessa estrutura como um modelo de um ímã cilíndrico de comprimento finito.
    \begin{enumerate}[label=(\alph*)]
        \item Encontre todas as correntes de magnetização do objeto. Em seguida, considerando que não há correntes livres, encontre o campo magnético sobre o eixo \(z\).
        \item Considere agora que um pequeno segundo objeto é posicionado acima do cilindro, exatamente sobre seu eixo de simetria, a uma distância \(d\) de sua extremidade superior, como mostra a figura abaixo. Esse pequeno objeto possui uma certa magnetização tal que possa ser caracterizado por um momento de dipolo \(\vetor{m} = m_0 \vetor{e}_z\). Calcule a força e o torque sobre esse pequeno objeto.
    \begin{center}
        \begin{tikzpicture}
            \draw (0,-3.5) -- (0,0.2);
            \fill[Subtext1] (-1,-3) rectangle (1,0);
            \fill[Overlay0] (0,0) ellipse (1 and 0.3);
            \fill[Subtext1] (0,-3) ellipse (1 and 0.3);

            \draw[stealth-stealth, thick] (-1.5,0) -- (-1.5,-3) node[midway,left] {\(L\)};

            \draw[stealth-stealth, thick] (-1.5,0) -- (-1.5,1.1) node[midway,left] {\(d\)};

            \draw[-stealth] (0,0) -- (0,2) node[above] {$z$};

            \fill[Red] (0,1.1) circle (1.5pt);
            \draw[-stealth, thick, Red] (0,0.85) -- (0,1.35) node[right] {\(\vetor{m}\)};

            \draw[densely dotted, thin, Sky] (0,1.15) -- ({5.5 + 1.5*cos(88)},{-0.5 + 1.5*sin(88)});
            \draw[densely dotted, thin, Sky] (0,1.05) -- ({5.5 - 1.5*cos(60)},{-0.5 - 1.5*sin(60)});

            \begin{scope}[shift={(5.5,-0.5)}, scale=0.75, every node/.style={scale=0.6}]
                \draw[Sky] (0,0) circle (2);

                \fill[Red!20] (0,0) circle (0.8);

                \fill[Text,opacity=0.3] (0,0) ellipse (1.2 and 0.3);
                \draw[thick] (0,0) ellipse (1.2 and 0.3);

                \draw[-stealth] plot [domain=280:330, samples=100] ({0.2 + 1.1*cos(\x)}, {-0.1 + 0.3*sin(\x)}) node[auto, below] {\(I_m\)};
                \draw[-stealth] plot [domain=-140:45, samples=100] ({1.2*cos(\x)}, {0.3*sin(\x)});
                \draw[-stealth] plot [domain=30:225, samples=100] ({1.2*cos(\x)}, {0.3*sin(\x)});

                \draw[-stealth, thick, Red] (0,0) -- (0,1.5);
                \node[right, Red] at (0,1.5) {\( \vetor{m} \)};
            \end{scope}

        \end{tikzpicture}
    \end{center}
        \item Suponha que, para todos os fins práticos, o objeto possa ser entendido como uma pequena espira circular de raio \(a \ll d, L, R\) conduzindo uma corrente de intensidade \(I_m\). Obtenha uma relação entre \(I_m,\) \(a,\) e \(m_0,\) e calcule o fluxo magnético \(\Phi_B\) que atravessa a pequena espira devido ao campo magnético gerado pelo ímã. Considere que esse campo magnético é aproximadamente uniforme sobre toda a área da pequena espira e é descrito pela expressão encontrada no item (a).
    \end{enumerate}
\end{exercício}
\begin{proof}[Resolução]
    Como a magnetização é uniforme, temos \(\nabla \times \vetor{M} = \vetor{0}\) em todo o espaço, isto é, \(\vetor{J}_M = \vetor{0}\). A superfície plana do cilindro tem normal paralela à direção da magnetização, portanto não há corrente superficial de magnetização nesta superfície. Na superfície lateral do cilindro temos
    \begin{equation*}
        \vetor{K}_M = \vetor{M} \times \vetor{e}_s = M \vetor{e}_\varphi
    \end{equation*}
    como a distribuição de corrente de magnetização. Considerando que não há correntes livres, temos
    \begin{align*}
        \vetor{B}(z \vetor{e}_z) &= \frac{\mu_0}{4\pi} \int_{-\frac{L}{2}}^{\frac{L}{2}}\dli{z'}\int_{0}^{2\pi} R \dli{\varphi'} \frac{M \vetor{e}_{\varphi'} \times \left[(z - z')\vetor{e}_z- R \vetor{e}_{s'}\right]}{\left[R^2 + (z - z')^2\right]^{\frac32}}\\
                                 &= \frac{\mu_0 M R}{4\pi} \int_{-\frac{L}{2}}^{\frac{L}{2}} \dli{z'} \int_0^{2\pi} \dli{\varphi'} \frac{(z - z')\vetor{e}_{s'} + R \vetor{e}_z}{\left[R^2 + (z - z')^2\right]^{\frac32}}\\
                                 &= \frac{\mu_0 M}{2} \int_{-\frac{L}{2}}^{\frac{L}{2}} \dli{z'}  \frac{R^2}{\left[R^2 + (z - z')^2\right]^{\frac32}}\vetor{e}_z
    \end{align*}
    pela lei de Biot-Savart. Com a mudança de variável \(z - z' = R \tan\psi,\) temos
    \begin{equation*}
        \vetor{B}(z\vetor{e}_z) = \frac{\mu_0 M}{2} \int_{\arctan\left(\frac{z - \frac{L}{2}}{R}\right)}^{\arctan\left(\frac{z + \frac{L}{2}}{R}\right)} \dli{\psi} \cos\psi \vetor{e}_z = \frac{\mu_0 M}{2}\left[\frac{z + \frac{L}{2}}{\sqrt{R^2 + \left(z + \frac{L}{2}\right)^2}} - \frac{z - \frac{L}{2}}{\sqrt{R^2 + \left(z - \frac{L}{2}\right)^2}}\right]\vetor{e}_z
    \end{equation*}
    como a expressão do campo magnético no eixo \(z\). Dessa forma, a força magnética sobre um dipolo \(\vetor{m} = m_0\vetor{e}_z\) na posição \(z = d\) é dada por
    \begin{align*}
        \vetor{F} &= \nabla \inner*{\vetor{m}}{\vetor{B}\left(\left(d+\frac{L}{2}\right)\vetor{e}_z\right)}\\
                  &= \frac{\mu_0m_0 M}{2}\left\{\frac{1}{\sqrt{R^2 + \left(z + \frac{L}{2}\right)^2}} - \frac{1}{\sqrt{R^2 + \left(z - \frac{L}{2}\right)^2}} - \frac{\left(z + \frac{L}{2}\right)^2}{\left[R^2 + \left(z + \frac{L}{2}\right)^2\right]^{\frac32}} + \frac{\left(z - \frac{L}{2}\right)^2}{\left[R^2 + \left(z - \frac{L}{2}\right)^2\right]^{\frac32}}\right\}_{z = d +\frac{L}{2}}\vetor{e}_z\\
                  &= \frac{\mu_0 m_0M R^2}{2}\left\{\frac{1}{\left[R^2 + \left(z + \frac{L}{2}\right)^2\right]^{\frac32}} - \frac{1}{\left[R^2 + \left(z - \frac{L}{2}\right)^2\right]^{\frac32}}\right\}_{z = d +\frac{L}{2}}\vetor{e}_z\\
                  &= \frac{\mu_0 m_0M R^2}{2}\left\{\frac{1}{\left[R^2 + \left(d + L\right)^2\right]^{\frac32}} - \frac{1}{\left(R^2 + d^2\right)^{\frac32}}\right\}\vetor{e}_z
    \end{align*}
    e o torque sobre o dipolo devido à interação com o campo é nulo.

    Por fim, considerando o objeto de momento de dipolo \(\vetor{m} = m_0 \vetor{e}_z\) como uma pequena espira circular de raio \(a\) com uma corrente \(I_m\), temos \(m_0 = I_m \pi a^2\) pelo \cref{ex:exercício1}. Assim, tomando o campo magnético uniforme face ao raio \(a\), temos
    \begin{equation*}
        \Phi_B = \frac{\mu_0 \pi a^2 M}{2}\left[\frac{d + L}{\sqrt{R^2 + \left(d + L\right)^2}} - \frac{d}{\sqrt{R^2 + d^2}}\right]
    \end{equation*}
    é o fluxo magnético pelo objeto.
\end{proof}
