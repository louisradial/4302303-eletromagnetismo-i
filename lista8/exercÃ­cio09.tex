\begin{exercício}{Ímã cilíndrico}{exercício9}
    Um cilindro maciço de raio \(R\) e comprimento \(L\) possui uma magnetização constante \(\vetor{M} = M \vetor{e}_z\), onde o eixo \(z\) e a origem coincidem, respectivamente, com o eixo de simetria e o centro do cilindro. Você pode pensar nessa estrutura como um modelo de um ímã cilíndrico de comprimento finito.
    \begin{enumerate}[label=(\alph*)]
        \item Encontre todas as correntes de magnetização do objeto. Em seguida, considerando que não há correntes livres, encontre o campo magnético sobre o eixo \(z\).
        \item Considere agora que um pequeno segundo objeto é posicionado acima do cilindro, exatamente sobre seu eixo de simetria, a uma distância \(d\) de sua extremidade superior, como mostra a figura abaixo. Esse pequeno objeto possui uma certa magnetização tal que o mesmo pode ser caracterizado por um momento de dipolo \(\vetor{m} = m_0 \vetor{e}_z\). Calcule a força e o torque sobre esse pequeno objeto.

    \begin{center}
        \begin{tikzpicture}
            \fill[Subtext1!70] (-1,-3) rectangle (1,0);
            \fill[Subtext1!50] (0,0) ellipse (1 and 0.3);
            \fill[Subtext1!70] (0,-3) ellipse (1 and 0.3);

            \draw[stealth-stealth, thick] (-1.5,0) -- (-1.5,-3) node[midway,left] {\(L\)};

            \draw[stealth-stealth, thick] (-1.5,0) -- (-1.5,1.1) node[midway,right] {\(d\)};

            \draw[dashed] (0,-3.5) -- (0,0.2);
            \draw[-stealth] (0,0) -- (0,2.5) node[above] {$z$};

            \fill[Red] (0,1.1) circle (1.5pt);
            \draw[-stealth, thick, Red] (0,0.85) -- (0,1.35) node[right] {\(\vetor{m}\)};

            \draw[densely dotted, thin, Sky] (0,1.15) -- ({5.5 + 1.5*cos(88)},{-0.5 + 1.5*sin(88)});
            \draw[densely dotted, thin, Sky] (0,1.05) -- ({5.5 - 1.5*cos(60)},{-0.5 - 1.5*sin(60)});

            \begin{scope}[shift={(5.5,-0.5)}, scale=0.75, every node/.style={scale=0.6}]
                \draw[Sky] (0,0) circle (2);

                \fill[Red!20] (0,0) circle (0.8);

                \fill[Text,opacity=0.3] (0,0) ellipse (1.2 and 0.3);
                \draw[thick] (0,0) ellipse (1.2 and 0.3);

                \draw[-stealth] plot [domain=280:330, samples=100] ({0.2 + 1.1*cos(\x)}, {-0.1 + 0.3*sin(\x)}) node[auto, below] {\(I_m\)};
                \draw[-stealth] plot [domain=-140:45, samples=100] ({1.2*cos(\x)}, {0.3*sin(\x)});
                \draw[-stealth] plot [domain=30:225, samples=100] ({1.2*cos(\x)}, {0.3*sin(\x)});

                \draw[-stealth, thick, Red] (0,0) -- (0,1.5);
                \node[right, Red] at (0,1.5) {\( \vetor{m} \)};
            \end{scope}

        \end{tikzpicture}
    \end{center}
        \item Suponha que, para todos os fins práticos, o objeto possa ser entendido como uma pequena espira circular de raio \(a \ll d, L, R\) conduzindo uma corrente de intensidade \(I_m\). Obtenha uma relação entre \(I_m,\) \(a,\) e \(m_0,\) e calcule o fluxo magnético \(\Phi_B\) que atravessa a pequena espira devido ao campo magnético gerado pelo ímã. Considere que esse campo magnético é aproximadamente uniforme sobre toda a área da pequena espira e é descrito pela expressão encontrada no item (a).
    \end{enumerate}

\end{exercício}
\begin{proof}[Resolução]
\end{proof}
