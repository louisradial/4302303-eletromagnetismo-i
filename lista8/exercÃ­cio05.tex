\begin{exercício}{Energia de interação entre dois dipolos magnéticos ideais}{exercício5}
    Mostre que a energia de interação entre dois dipolos magnéticos ideais \(\vetor{m}_1\) e \(\vetor{m}_2\) pode ser escrita como
    \begin{equation*}
        U = \frac{\mu_0}{4\pi \norm{\vetor{\x}_2 - \vetor{\x}_1}^3}\left[\inner{\vetor{m}_1}{\vetor{m}_2} - 3\inner*{\vetor{m}_1}{\frac{\vetor{\x}_2 - \vetor{\x}_1}{\norm{\vetor{\x}_2 - \vetor{\x}_1}}}\inner*{\vetor{m}_2}{\frac{\vetor{\x}_2 - \vetor{\x}_1}{\norm{\vetor{\x}_2 - \vetor{\x}_1}}}\right] - \frac{2\mu_0}{3}\inner{\vetor{m}_1}{\vetor{m}_2} \delta(\vetor{\x}_2 - \vetor{\x}_1)
    \end{equation*}
    em que \(\vetor{\x}_i\) é a posição do dipolo de momento \(\vetor{m}_i\).
\end{exercício}
\begin{proof}[Resolução]

\end{proof}
