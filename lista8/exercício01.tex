\begin{exercício}{Momento de dipolo}{exercício1}
    No contexto da expansão multipolar, o potencial vetor do termo de dipolo é dado por
    \begin{equation*}
        \vetor{A}_{\mathrm{dip}}(\vetor{\x}) = \frac{\mu_0}{4\pi} \frac{\vetor{m}\times\vetor{\x}}{\norm{\vetor{\x}}^3}.
    \end{equation*}
    Faça um resumo das três diferentes expressões para o momento de dipolo \(\vetor{m}\), a depender do tipo de distribuição de corrente. Para o último caso, utilize que o vetor área de uma superfície \(S\) cercada por um acurva fechada \(\Gamma\) pode ser escrito como
    \begin{equation*}
        \vetor{a} = \int_S \dln2{\x'} \vetor{n'} =-\frac12 \oint_{\Gamma} \dli{\vetor{\ell}} \times \vetor{\x}
    \end{equation*}
    para mostrar que podemos escrever o momento de dipolo \(\vetor{m}\) como \(I\vetor{a}\).
\end{exercício}
\begin{proof}[Resolução]

\end{proof}
