\begin{exercício}{Campo \(\vetor{H}\) para um dipolo magnético ideal}{exercício7}
    Um dipolo magnético ideal do tipo \(\vetor{m̀} = m \vetor{e}_z\) está localizado na origem do sistema de coordenadas. Considere que há vácuo em todo o espaço para \(r > 0\).
    \begin{enumerate}[label=(\alph*)]
        \item Obtenha \(\vetor{H}\) para \(r > 0\).
        \item Mostre que \(\nabla \times \vetor{H} = 0\) para todo \(r > 0\).
        \item Determine um potencial escalar \(\phi_M\) para este problema.
        \item O que mudaria caso o espaço todo fosse inteiramente preenchido por um material magnético linear de permeabilidade \(\mu\)?
    \end{enumerate}
\end{exercício}
\begin{proof}[Resolução]

\end{proof}
