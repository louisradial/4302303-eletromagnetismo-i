\begin{lemma}{Corolário do laplaciano do recíproco da norma do vetor deslocamento}{zangwill122}
    Para todos \(\vetor{\x},\vetor{\x'} \in \mathbb{R}^3\), vale
    \begin{equation*}
        \diffp*{\diffp*{\frac{1}{\norm{\D}}}{\x_j}}{\x_i} = \frac{3\inner{\vetor{e}_i}{\D}\inner{\vetor{e}_j}{\D} - \delta_{ij}\norm{\D}^2}{\norm{\D}^5} - \frac{4\pi}{3}\delta_{ij}\delta(\D)
    \end{equation*}
    com \(i,j \in \set{1,2,3}\).
\end{lemma}
\begin{proof}
    Para \(\vetor{\x} \in \mathbb{R}^3 \setminus \set{\vetor{\x'}}\), temos
    \begin{align*}
        \diffp*{\diffp*{\frac{1}{\norm{\D}}}{\x_j}}{\x_i} &= \diffp*{\diffp*{\left(\sum_{k = 1}^3 \inner{\vetor{e}_k}{\D}^2\right)^{-\frac12}}{\x_j}}{\x_i}\\
                                                          &= -\diffp*{\left[\inner{\vetor{e}_j}{\D}\left(\sum_{k = 1}^3 \inner{\vetor{e}_k}{\D}^2\right)^{-\frac32}\right]}{\x_i}\\
                                                          &=\frac{3\inner{\vetor{e}_i}{\D}\inner{\vetor{e}_j}{\D} - \delta_{ij}\norm{\D}^2}{\norm{\D}^5}
    \end{align*}
    para quaisquer \(i,j \in \set{1,2,3}\).

    Consideramos o ansatz para todos \(\vetor{\x}, \vetor{\x'} \in \mathbb{R}^3\)
    \begin{equation*}
        \diffp*{\diffp*{\frac{1}{\norm{\D}}}{\x_j}}{\x_i} = \frac{3\inner{\vetor{e}_i}{\D}\inner{\vetor{e}_j}{\D} - \delta_{ij}\norm{\D}^2}{\norm{\D}^5} - \alpha \delta_{ij}\delta(\D),
    \end{equation*}
    onde \(\alpha\) é constante. Temos
    \begin{align*}
        -4\pi \delta(\D) = \nabla^2\left(\frac{1}{\norm{\D}}\right)
        &= \sum_{i = 1}^3 \diffp*[2]{\frac{1}{\norm{\D}}}{\x_i}\\
        &= \sum_{i = 1}^3 \left[\frac{3\inner{\vetor{e}_i}{\D}^2 - \norm{\D}^2}{\norm{\D}^5} - \alpha \delta(\D)\right]\\
        &= -3 \alpha \delta(\D),
    \end{align*}
    portanto \(\alpha = \frac{4\pi}{3}\) e concluímos a demonstração.
\end{proof}
\begin{exercício}{Campo magnético associado ao potencial de dipolo}{exercício3}
    Calcule o campo magnético associado ao termo de dipolo \(\vetor{A}_\mathrm{dip}\).
\end{exercício}
\begin{proof}[Resolução]
    Temos
    \begin{align*}
        \nabla \times \vetor{A}_\mathrm{dip}(\vetor{\x})
        &= \frac{\mu_0}{4\pi}\nabla \times \left[\nabla \left(\frac1{\norm{\vetor{\x}}}\right)\times \vetor{m}\right]\\
        &= \frac{\mu_0}{4\pi}\left[- \nabla^2\left(\frac{1}{\norm{\vetor{\x}}}\right)\vetor{m} + (\vetor{m} \cdot \nabla)\nabla\left(\frac1{\norm{\vetor{\x}}}\right)\right]\\
        &= \frac{\mu_0}{4\pi}\left[4\pi\vetor{m} \delta(\vetor{\x}) + m_i\partial_i \vetor{e}_j \partial_j\left(\frac{1}{\norm{\vetor{\x}}}\right)\right]\\
        &= \frac{\mu_0}{4\pi}\left\{4\pi \vetor{m} \delta(\vetor{\x}) + m_i \vetor{e}_j \left[\frac{3\inner{\vetor{e}_i}{\vetor{\x}}\inner{\vetor{e}_j}{\vetor{\x}} - \delta_{ij}\norm{\vetor{\x}}^2}{\norm{\vetor{\x}}^5} - \frac{4\pi}{3}\delta_{ij}\delta(\vetor{\x})\right]\right\}
    \end{align*}
    pelo \cref{lem:zangwill122}. Assim, o campo magnético associado ao termo de dipolo é
    \begin{align*}
        \vetor{B}_{\mathrm{dip}}(\vetor{\x}) &= \frac{\mu_0}{4\pi}\left[\frac{8\pi}{3}\vetor{m} \delta(\vetor{\x}) + \frac{3\inner{\vetor{m}}{\vetor{\x}}\inner{\vetor{e}_j}{\vetor{\x}}\vetor{e}_j - \norm{\vetor{\x}}^2 \vetor{m}}{\norm{\vetor{\x}}^5}\right]\\
                                             &= \frac{2\mu_0}{3}\vetor{m} \delta(\vetor{\x}) + \frac{3\inner{\vetor{m}}{\vetor{\x}}\vetor{\x} - \norm{\vetor{\x}}^2\vetor{m}}{\norm{\vetor{\x}}^5}.
    \end{align*}
    A não ser que seja o campo magnético de um dipolo ideal, utilizamos a expansão multipolar com \(\vetor{\x}\) fora da distribuição de corrente, de forma que o termo com o delta de Dirac possa ser desconsiderado para aplicações que não envolvam integrais de volume em regiões que contenham a distribuição de corrente.
\end{proof}
