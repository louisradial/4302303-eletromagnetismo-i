\begin{exercício}{Dipolo magnético na presença de um campo uniforme}{exercício4}
    Em uma região do espaço encontra-se um campo magnético uniforme, \(\vetor{B} = B_0 \vetor{e}_z\). Adicionalmente, um dipolo magnético do tipo \(\vetor{m} = -m_0 \vetor{e}_z\), com \(m_0 > 0\), é colocado na origem do sistema de coordenadas. Mostre que existe uma superfície esférica, centrada na origem pela qual não passam linhas de campo magnético. Encontre o raio dessa esfera e faça um esboço das linhas de campo, internas e externas a essa superfície. Assuma que o campo magnético gerado pelo dipolo é dado pela expressão do problema anterior.
\end{exercício}
\begin{proof}[Resolução]

\end{proof}
