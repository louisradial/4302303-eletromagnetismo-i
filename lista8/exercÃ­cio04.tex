\begin{exercício}{Dipolo magnético na presença de um campo uniforme}{exercício4}
    Em uma região do espaço encontra-se um campo magnético uniforme, \(\vetor{B}_0 = B_0 \vetor{e}_z\). Adicionalmente, um dipolo magnético do tipo \(\vetor{m} = -m_0 \vetor{e}_z\), com \(m_0 > 0\), é colocado na origem do sistema de coordenadas. Mostre que existe uma superfície esférica, centrada na origem pela qual não passam linhas de campo magnético. Encontre o raio dessa esfera e faça um esboço das linhas de campo, internas e externas a essa superfície. Assuma que o campo magnético gerado pelo dipolo é dado pela expressão do problema anterior.
\end{exercício}
\begin{proof}[Resolução]
    Pelo \cref{ex:exercício3}, o campo total é dado por
    \begin{equation*}
        \vetor{B}(\vetor{\x}) = B_0 \vetor{e}_z + \frac{\mu_0}{4\pi \norm{\vetor{\x}}^5} \left[3 \inner{\vetor{\x}}{\vetor{m}}\vetor{\x} - \norm{\vetor{\x}}^2 \vetor{m}\right] = \left(B_0+\frac{\mu_0m_0}{4\pi \norm{\vetor{\x}}^3}\right)\vetor{e}_z - \frac{3\mu_0\inner{\vetor{\x}}{\vetor{e}_z}m_0 \vetor{\x}}{4\pi \norm{\vetor{\x}}^5}.
    \end{equation*}
    Assim, o fluxo de campo magnético pela superfície esférica \(\Omega_R\) de raio \(R\) centrada na origem é
    \begin{align*}
        \Phi(R) = \int_{\Omega_R} \dln3\x \vetor{n} \cdot \vetor{B}(\vetor{\x})
        &= \int_{0}^{\pi} R \dli{\theta} \int_0^{2\pi}R \sin\theta \dli{\varphi} \left[\left(B_0 + \frac{\mu_0 m_0}{4\pi R^3}\right) \cos\theta - \frac{3\mu_0m_0 \cos\theta}{4\pi R^3}\right]\\
        &= \frac{1}{2R} \int_0^\pi \sin\theta \dli{\theta} \left[4\pi R^3 B_0 - 2\mu_0 m_0\right]\cos\theta\\
        &= \frac{1}{2R} \left(4\pi R^3 B_0 - 2\mu_0 m_0\right).
    \end{align*}
    Dessa forma, para \(R_0 = \left(\frac{\mu_0 m_0}{2\pi B_0}\right)^{\frac13}\), temos \(\Phi(R_0) = 0\).

    \todo[Linhas de campo.]
\end{proof}
