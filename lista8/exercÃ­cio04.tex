\begin{exercício}{Dipolo magnético na presença de um campo uniforme}{exercício4}
    Em uma região do espaço encontra-se um campo magnético uniforme, \(\vetor{B}_0 = B_0 \vetor{e}_z\). Adicionalmente, um dipolo magnético do tipo \(\vetor{m} = -m_0 \vetor{e}_z\), com \(m_0 > 0\), é colocado na origem do sistema de coordenadas. Mostre que existe uma superfície esférica, centrada na origem pela qual não passam linhas de campo magnético. Encontre o raio dessa esfera e faça um esboço das linhas de campo, internas e externas a essa superfície. Assuma que o campo magnético gerado pelo dipolo é dado pela expressão do problema anterior.
\end{exercício}
\begin{proof}[Resolução]
    Pelo \cref{ex:exercício3}, o campo total é dado por
    \begin{align*}
        \vetor{B}(\vetor{\x}) &= B_0 \vetor{e}_z + \frac{2\mu_0}{3}\vetor{m} \delta(\vetor{\x}) +  \frac{\mu_0}{4\pi \norm{\vetor{\x}}^5} \left[3 \inner{\vetor{\x}}{\vetor{m}}\vetor{\x} - \norm{\vetor{\x}}^2 \vetor{m}\right]\\
                              &= \left[B_0+\frac{\mu_0m_0}{4\pi \norm{\vetor{\x}}^3}-\frac{2\mu_0 m_0}{3} \delta(\vetor{\x})\right]\vetor{e}_z - \frac{3\mu_0\inner{\vetor{\x}}{\vetor{e}_z}m_0 \vetor{\x}}{4\pi \norm{\vetor{\x}}^5}.
    \end{align*}
    Dessa forma, para \(\vetor{\x} \neq \vetor{0}\), temos
    \begin{equation*}
        \frac{\vetor{\x}}{\norm{\vetor{\x}}} \cdot \vetor{B}(\vetor{\x}) = \left(B_0 - \frac{\mu_0 m_0}{2\pi \norm{\vetor{\x}}^3}\right)\inner{\vetor{e}_z}{\vetor{\x}},
    \end{equation*}
    isto é, não há fluxo de campo magnético por trechos da superfície esférica de raio \(R = \left(\frac{\mu_0 m_0}{2\pi B_0}\right)^{\frac13}\) centrada na origem.

    \todo[Linhas de campo.]
\end{proof}
