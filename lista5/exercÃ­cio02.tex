\begin{exercício}{Adição de harmônicos esféricos}{exercício2}
    A solução geral para a equação de Laplace em coordenadas esféricas é dada por
    \begin{equation*}
        \phi(r, \theta, \varphi) = \sum_{\ell = 0}^\infty \sum_{m = -\ell}^\ell \left(A_{\ell m}r^\ell + \frac{B_{\ell m}}{r^{\ell + 1}}\right) Y_{\ell m}(\theta, \varphi),
    \end{equation*}
    onde
    \begin{equation*}
        Y_{\ell m}(\theta, \varphi) = \sqrt{\frac{(2\ell + 1)(\ell - \abs{m})!}{4\pi \left(\ell + \abs{m}\right)!}} P_\ell^m(\cos\theta)e^{im\varphi}
    \end{equation*}
    são os harmônicos esféricos e \(P_\ell^m\) denota o polinômio associado de Legendre. Alguns dos primeiros harmônicos esféricos são
    \begin{equation*}
        Y_{00} = \frac{1}{\sqrt{4\pi}},\quad
        Y_{1-1} = \sqrt{\frac{3}{8\pi}}\sin\theta e^{-i\phi},\quad
        Y_{11} = -\sqrt{\frac{3}{8\pi}}\sin\theta e^{i\phi},\quad\text{e}\quad
        Y_{10} = \sqrt{\frac{3}{4\pi}} \cos\theta.
    \end{equation*}

    \begin{center}

        \tdplotsetmaincoords{60}{120}
        \begin{tikzpicture}[scale=1.2,tdplot_main_coords]

    % Axes
    \draw[thick,->] (0,0,0) -- (0,3,0) node[anchor=north east] {\(y\)};  % y axis
    \draw[thick,->] (0,0,0) -- (0,0,3) node[anchor=south east] {\(z\)};  % z axis
    \draw[thick,->] (0,0,0) -- (3,0,0) node[anchor=south] {\(x\)};  % x axis

    % vector 1
    \pgfmathsetmacro{\ax}{1}
    \pgfmathsetmacro{\ay}{2}
    \pgfmathsetmacro{\az}{1.5}
    \draw[very thick,->] (0,0,0) -- (\ax,\ay,\az) node[anchor=west]{\(\vetor{\x}\)};
    % Dashed projection onto xy-plane
    \draw[dashed] (\ax,\ay,0) -- (\ax,\ay,\az);  % Projection of r onto xy-plane
    \draw[dashed] (0,0,0) -- (\ax,\ay,0);   % Origin to projection of r

    % vector 2
    \pgfmathsetmacro{\bx}{1}
    \pgfmathsetmacro{\by}{2}
    \pgfmathsetmacro{\bz}{2}
    \draw[very thick,->] (0,0,0) -- (\bx,\by,\bz) node[anchor=west]{\(\vetor{\x'}\)};
    % Dashed projection onto xy-plane
    \draw[dashed] (\bx,\by,0) -- (\bx,\by,\bz);  % Projection of r onto xy-plane
    \draw[dashed] (0,0,0) -- (\bx,\by,0);   % Origin to projection of r
    %Polar angle theta
    % \tdplotsetthetaplanecoords{45}
    \tdplotdefinepoints(0,0,0)(0,0,0.7071)(0.3,0.4,0.5)
    \tdplotdrawpolytopearc[->]{0.7}{anchor=south west}{$\theta$}
    % \draw[->] (0,0.3,0) arc[start angle=90, end angle=20, radius=0.3] node[midway, above] {$\theta$};
    %
    % Azimuthal angle phi
    % \tdplotsetthetaplanecoords{30}
    \tdplotdefinepoints(0,0,0)(0.5,0,0)(0.3,0.4,0)
    \tdplotdrawpolytopearc[->]{0.5}{anchor=north}{$\varphi$}
    % \draw[->] (0,0,0) arc[start angle=0, end angle=45, radius=0.4] node[midway, below] {$\varphi$};

\end{tikzpicture}

    \end{center}
\end{exercício}
\begin{proof}[Resolução]

\end{proof}
