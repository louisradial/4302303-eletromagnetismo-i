\begin{exercício}{Unidades de medida aos momentos de monopolo, dipolo e quadrupolo}{exercício3}
    No contexto da expansão multipolar do potencial elétrico, quais são as unidades de medida associadas aos momentos de monopolo, dipolo, e quadrupolo, no SI?
\end{exercício}
\begin{proof}[Resolução]
    O momento de monopolo é a carga total de um sistema, portanto sua unidade é \(\unit{\coulomb} = \unit{\ampere \second}\) no SI. O momento de dipolo é dado por
    \begin{equation*}
        \vetor{p} = \int_{\mathbb{R}^3}  \dln3{\x'} \vetor{\x'}\rho(\vetor{\x'}),
    \end{equation*}
    portanto sua unidade é \(\unit{\coulomb \meter} = \unit{\ampere \meter \second}\) no SI. Os momentos de quadrupolo são dados por
    \begin{equation*}
        Q_{ij} = \int_{\mathbb{R}^3} \dln3{\x'} \left(3 \x'_i \x'_j - \norm{\vetor{\x'}}^2 \delta_{ij}\right)\rho(\vetor{\x'}),
    \end{equation*}
    portanto sua unidade é \(\unit{\coulomb \meter^2} = \unit{\ampere \meter^2 \second}\) no SI.
\end{proof}
