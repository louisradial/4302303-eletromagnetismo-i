\begin{exercício}{Esfera condutora aterrada com um dipolo ideal em seu interior}{exercício7}
    Uma casca esférica condutora de raio \(R\) encontra-se aterrada. No centro da casca há um dipolo ideal \(\vetor{p}\) alinhado com o eixo \(z\), no sentido positivo. Encontre o potencial elétrico dentro e fora da casca. Em seguida, obtenha a densidade de carga induzida na casca e calcule a carga total induzida. O que mudaria na sua resposta caso a esfera estivesse mantida a um potencial \(\phi_0\) ao invés de aterrada?
\end{exercício}
\begin{proof}[Resolução]
    Sejam \(\Omega = \setc{\vetor{\x} \in \mathbb{R}^3}{\norm{\vetor{\x}} > R}\) e \(\Sigma = \setc{\vetor{\x} \in \mathbb{R}^3}{\norm{\vetor{\x}} < R}\) as regiões exterior e interior à superfície esférica \(\partial \Sigma = \setc{\vetor{\x} \in \mathbb{R}^3}{\norm{\vetor{\x}} = R}.\) Como o potencial satisfaz a equação de Laplace em \(\Omega\) com condições de contorno \(\phi(\vetor{\x}) = 0\) para todo \(\vetor{\x} \in \partial \Omega\), segue que \(\phi(\vetor{\x}) = 0\) para todo \(\vetor{\x} \in \Omega.\)

    Em \(\Sigma\) o potencial satisfaz a equação de Laplace com condições de contorno dadas por \(\phi(\vetor{\x}) = 0\) para todo \(\vetor{\x} \in \partial \Sigma\) e \(\phi(\vetor{\x}) \sim \frac{\inner{\vetor{p}}{\vetor{\x}}}{4\pi \epsilon_0 \norm{\vetor{\x}}^3}\) para \(\norm{\vetor{\x}} \ll R\). Como há simetria azimutal, podemos escrever
    \begin{equation*}
        \phi(r,\theta) = \sum_{\ell = 0}^\infty \left(A_\ell r^\ell + \frac{B_{\ell}}{r^{\ell + 1}}\right) P_\ell(\cos\theta)
    \end{equation*}
    como a solução geral da equação de Laplace. Como \(\phi(R, \theta) = 0\), devemos ter
    \begin{equation*}
        A_\ell R^\ell = - \frac{B_{\ell}}{R^{\ell + 1}} \implies B_{\ell} = -R^{2\ell + 1} A_{\ell},
    \end{equation*}
    logo podemos escrever
    \begin{equation*}
        \phi(r, \theta) = \sum_{\ell = 0}^\infty \left[\left(\frac{r}{R}\right)^{\ell} - \left(\frac{R}{r}\right)^{\ell + 1}\right] R^\ell A_\ell P_{\ell}(\cos\theta).
    \end{equation*}
    No limite em que \(r \ll R\) temos
    \begin{equation*}
        \phi(r, \theta) \sim \frac{p \cos\theta}{4\pi \epsilon_0 r^2} \implies -\sum_{\ell = 0}^\infty \left(\frac{R}{r}\right)^{\ell + 1}R^\ell A_\ell P_{\ell}(\cos\theta) = \frac{p \cos\theta}{4\pi \epsilon_0 r^2},
    \end{equation*}
    portanto concluímos que
    \begin{equation*}
        R^\ell A_\ell = -\frac{p}{4\pi \epsilon_0 R^{\ell + 1}}\delta_{\ell 1}
    \end{equation*}
    para todo \(\ell \in \mathbb{N}_0\). Isto é,
    \begin{equation*}
        \phi(\vetor{\x}) = \begin{cases}
            0,&\text{se } \vetor{\x} \in \Omega \cup \partial \Sigma\\
            \displaystyle \frac{\inner{\vetor{p}}{\vetor{\x}}}{4\pi \epsilon_0 R^2 \norm{\vetor{\x}}}\left[\left(\frac{R}{\norm{\vetor{\x}}}\right)^2 - \frac{\norm{\vetor{\x}}}{R}\right],&\text{se }\vetor{\x} \in \Sigma
        \end{cases}
    \end{equation*}
    é o potencial em todo o espaço.

    Desse modo, a densidade de carga induzida na casca é dada por
    \begin{equation*}
        \sigma(\theta) = \epsilon_0\left(-\diffp{\phi}{r}[r = R^+] + \diffp{\phi}{r}[r = R^-]\right) = \frac{p\cos\theta}{4\pi R^2} \left[-\frac{R^2}{R^3} - \frac{1}{R}\right] = -\frac{p \cos\theta}{2\pi R^3}
    \end{equation*}
    e então
    \begin{equation*}
        q = \int_{\partial \Sigma} \dln2{\x} \sigma = -\frac{p}{R}\int_0^\pi \dli{\theta} \sin\theta \cos\theta = 0
    \end{equation*}
    é a carga total induzida na casca esférica.

    \todo[Casca esférica a potencial constante.]
\end{proof}
