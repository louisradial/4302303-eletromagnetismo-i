\begin{exercício}{Placas condutoras}{exercício8}
    Duas placas condutoras estão dispostas como mostrado no arranjo da figura abaixo. O eixo \(z\) é perpendicular ao plano da figura e as placas são infinitas na direção \(z\). O ângulo que o plano das placas faz com o eixo \(x\) é \(\varphi_0\) para a placa superior e \(-\varphi_0\) para a placa inferior. Além disso, a placa superior está mantida a um potencial \(\phi_0\) enquanto que a placa inferior está mantida a um potencial \(-\phi_0\).
    \begin{center}

        \begin{tikzpicture}[scale = 0.7]
            % Axes
            \draw[->] (-1,0) -- (5,0) node[right] {\(x\)};
            \draw[->] (0,-3) -- (0,3) node[above] {\(y\)};

            % Lines at phi_0
            \draw[thick] ({1.0*cos(30)},{1.0*sin(30)}) -- ({5*cos(30)},{5*sin(30)}) node[anchor=south east] {\(\phi_0\)};
            \draw[thick] ({1.0*cos(-30)},{1.0*sin(-30)}) -- ({5*cos(-30)},{5*sin(-30)}) node[anchor=north east] {\(-\phi_0\)};

            % Dotted lines for angle indicators
            \draw[dotted] (0,0) -- ({1.0*cos(30)},{1.0*sin(30)});
            \draw[dotted] (0,0) -- ({1.0*cos(-30)},{1.0*sin(-30)});
            \draw (1,0) arc (0:30:1) node[midway, right] {\(\varphi_0\)};
        \end{tikzpicture}
    \end{center}
    \begin{enumerate}[label=(\alph*)]
        \item Desprezando-se efeitos de borda (suponha que a largura das placas seja suficientemente grande para isso), encontre o potencial na região entre as placas, isto é, para \(\varphi \in [-\varphi_0, \varphi_0]\). Utilize coordenadas cilíndricas e assuma que \(\phi = \phi(\varphi)\), ou seja, resolva a equação de Laplace para um potencial que só dependa da coordenada \(\varphi\). Repare que isso é consistente com linhas de campo elétrico que coincidem com arcos de circunferência que \enquote{saem} da placa superior e \enquote{entram} na placa inferior. Encontre o campo elétrico a partir do potencial.
        \item Agora, considere que um dipolo ideal do tipo
            \begin{equation*}
                \vetor{p} = p\left(\frac{\vetor{e}_1 + \vetor{e}_2}{\sqrt{2}}\right)
            \end{equation*}
            seja colocado na região entre as placas, na posição \(\vetor{\x}_p = d\vetor{e}_1\). A partir do resultado do item anterior, calcule a força e o torque sobre o dipolo. Calcule a energia de interação entre o dipolo e o campo gerado pelas placas.
    \end{enumerate}
\end{exercício}
\begin{proof}[Resolução]
    Desprezando-se efeitos de borda, o potencial na região entre as placas assume a forma \(\phi = \phi(\varphi)\) e satisfaz a equação de Laplace com condições de contorno \(\phi(\pm\varphi_0) = \pm \phi_0\). Assim, temos
    \begin{equation*}
        \nabla^2\phi = \frac{1}{s^2}\diffp[2]{\phi}{\varphi} = 0 \implies \phi(\varphi) = A\varphi + B,
    \end{equation*}
    para constantes \(A, B \in \mathbb{R}\). Das condições de contorno obtemos
    \begin{equation*}
        \begin{cases}
            A \varphi_0 + B = \phi_0\\
            -A \varphi_0 + B = -\phi_0
        \end{cases} \implies B = 0\quad\text{e}\quad A = \frac{\phi_0}{\varphi_0},
    \end{equation*}
    portanto
    \begin{equation*}
        \phi(\varphi) = \frac{\varphi}{\varphi_0}\phi_0
    \end{equation*}
    é o potencial na região entre as placas. Portanto, nessa região o campo elétrico é dado por
    \begin{equation*}
        \vetor{E}(s, \varphi) = -\frac{1}{s}\diffp{\phi}{\varphi}\vetor{e}_\varphi = -\frac{\phi_0}{\varphi_0 s} \vetor{e}_\varphi.
    \end{equation*}

    Se um dipolo ideal \(\vetor{p}\) é colocado na região entre as placas na posição \(\vetor{\x}_p\), a força sobre o dipolo é
    \begin{align*}
        \vetor{F} = \left(\vetor{p}\cdot \nabla\right)\vetor{E}(\vetor{\x}_p)
        &= \frac{p\phi_0}{\sqrt{2}\varphi_0}\left(\diffp{}{x} + \diffp{}{y}\right)_{(x,y) = (d,0)}\left[\frac{\sin\varphi \vetor{e}_1 - \cos\varphi\vetor{e}_2}{s}\right]\\
        &=\frac{p\phi_0}{\sqrt{2}\varphi_0}\left(\diffp{}{x} + \diffp{}{y}\right)_{(x,y)=(d,0)}\left[\frac{y \vetor{e}_1 - x\vetor{e}_2}{x^2 + y^2}\right]\\
        &=  \frac{p \phi_0}{\sqrt{2}\varphi_0} \left[\frac{x^2 - 2xy - y^2}{(x^2 + y^2)^2}\vetor{e}_1 + \frac{x^2 + 2xy - y^2}{(x^2 + y^2)^2}\vetor{e}_2\right]_{(x,y) = (d,0)}\\
        &=  \frac{p \phi_0}{\sqrt{2} \varphi_0d^2}\left(\vetor{e}_1 + \vetor{e}_2\right) = \frac{\phi_0\vetor{p}}{\varphi_0 d^2}
    \end{align*}
    e o torque sobre o dipolo é
    \begin{align*}
        \vetor{\tau} = \vetor{p} \times \vetor{E}(\vetor{\x}_p) &= \left[\frac{p}{\sqrt{2}}\left(\vetor{e}_1 + \vetor{e}_2\right)\right] \times \left(-\frac{\phi_0}{\varphi_0 d}\vetor{e}_2\right)\\
                                                                &= - \frac{p\phi_0}{\sqrt{2} \varphi_0 d}\vetor{e}_3.
    \end{align*}
    Temos também
    \begin{equation*}
        U_\mathrm{int} = - \vetor{p} \cdot \vetor{E}(\vetor{\x}_p) = \frac{p\phi_0}{\varphi_0\sqrt{2}d}
    \end{equation*}
    como a energia de interação entre o dipolo e o campo gerado pelas placas.
\end{proof}
