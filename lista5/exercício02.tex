\begin{exercício}{Adição de harmônicos esféricos}{exercício2}
    A solução geral para a equação de Laplace em coordenadas esféricas é dada por
    \begin{equation*}
        \phi(r, \theta, \varphi) = \sum_{\ell = 0}^\infty \sum_{m = -\ell}^\ell \left(A_{\ell m}r^\ell + \frac{B_{\ell m}}{r^{\ell + 1}}\right) Y_{\ell m}(\theta, \varphi),
    \end{equation*}
    onde
    \begin{equation*}
        Y_{\ell m}(\theta, \varphi) = \sqrt{\frac{(2\ell + 1)(\ell - \abs{m})!}{4\pi \left(\ell + \abs{m}\right)!}} P_\ell^m(\cos\theta)e^{im\varphi}
    \end{equation*}
    são os harmônicos esféricos e \(P_\ell^m\) denota o polinômio associado de Legendre. Alguns dos primeiros harmônicos esféricos são
    \begin{equation*}
        Y_{00} = \frac{1}{\sqrt{4\pi}},\quad
        Y_{1-1} = \sqrt{\frac{3}{8\pi}}\sin\theta e^{-i\phi},\quad
        Y_{11} = -\sqrt{\frac{3}{8\pi}}\sin\theta e^{i\phi},\quad\text{e}\quad
        Y_{10} = \sqrt{\frac{3}{4\pi}} \cos\theta.
    \end{equation*}

    Considere dois vetores \(\vetor{\x}\) e \(\vetor{\x'}\) com um ângulo planar \(\gamma\) entre eles. Fixando eixos coordenados, sejam \(\theta, \theta'\) os ângulos polares e \(\varphi, \varphi'\) os ângulos azimutais de cada vetor, como na figura abaixo.
    \begin{center}

        \tdplotsetmaincoords{60}{120}
        \begin{tikzpicture}[scale=1.2,tdplot_main_coords]

    % Axes
    \draw[thick,->] (0,0,0) -- (0,3,0) node[anchor=north east] {\(y\)};  % y axis
    \draw[thick,->] (0,0,0) -- (0,0,3) node[anchor=south east] {\(z\)};  % z axis
    \draw[thick,->] (0,0,0) -- (3,0,0) node[anchor=south] {\(x\)};  % x axis

    % vector 1
    \pgfmathsetmacro{\ax}{1.0};
    \pgfmathsetmacro{\ay}{3.0};
    \pgfmathsetmacro{\az}{3.5};
    \pgfmathsetmacro{\ar}{(\ax^2 + \ay^2 + \az^2)^0.5};
    \pgfmathsetmacro{\as}{(\ax^2 + \ay^2)^0.5};
    \pgfmathsetmacro{\aphi}{atan2(\ay,\ax)};
    \draw[very thick,-stealth,Mauve] (0,0,0) -- (\ax,\ay,\az) node[anchor=west]{\(\vetor{\x}\)};
    % dashed projection onto xy-plane
    \draw[dashed, Mauve] (\ax,\ay,0) -- (\ax,\ay,\az);  % Projection of r onto xy-plane
    \draw[dashed,Mauve] (0,0,0) -- (\ax,\ay,0);   % Origin to projection of r
    % polar angle
    \tdplotdefinepoints(0,0,0)(0,0,\ar)(\ax,\ay,\az)
    \tdplotdrawpolytopearc[-,Mauve]{\ar/5}{anchor=south west}{\color{Mauve}\(\theta\)};
    % azimuthal angle
    \tdplotdrawarc[tdplot_main_coords, Mauve]{(0,0,0)}{\as/5}{0}{\aphi}{anchor=north}{\(\varphi\)}

    % vector 2
    \pgfmathsetmacro{\bx}{3.5};
    \pgfmathsetmacro{\by}{1.0};
    \pgfmathsetmacro{\bz}{3.0};
    \pgfmathsetmacro{\br}{(\bx^2 + \by^2 + \bz^2)^0.5};
    \pgfmathsetmacro{\bs}{(\bx^2 + \by^2)^0.5};
    \pgfmathsetmacro{\bphi}{atan2(\by,\bx)};
    \draw[very thick,-stealth,Pink] (0,0,0) -- (\bx,\by,\bz) node[anchor=south east]{\(\vetor{\x'}\)};
    % dashed projection onto xy-plane
    \draw[dashed,Pink] (\bx,\by,0) -- (\bx,\by,\bz);  % Projection of r onto xy-plane
    \draw[dashed,Pink] (0,0,0) -- (\bx,\by,0);   % Origin to projection of r
    % polar angle
    \tdplotdefinepoints(0,0,0)(0,0,\br)(\bx,\by,\bz)
    \tdplotdrawpolytopearc[-,Pink]{\br/2.5}{anchor=south east}{\color{Pink}$\theta'$};
    % azimuthal angle
    \tdplotdrawarc[tdplot_main_coords,Pink]{(0,0,0)}{\bs/2}{0}{\bphi}{anchor=north}{\(\varphi'\)};

    % planar angle
    \tdplotdefinepoints(0,0,0)(3.5,1.0,3.0)(1.0,3.0,3.5)
    \tdplotdrawpolytopearc[-]{4}{anchor=south west}{$\gamma$};
\end{tikzpicture}
    \end{center}
    O teorema da adição dos harmônicos esféricos estabelece que
    \begin{equation*}
        P_\ell(\cos \gamma) = \frac{4\pi}{2\ell + 1} \sum_{m = -\ell}^{\ell} \conj{Y_{\ell m}}(\theta', \varphi')Y_{\ell m}(\theta,\varphi)
    \end{equation*}
    para todo \(\ell \in \mathbb{N}_0\).
    \begin{enumerate}[label=(\alph*)]
        \item Verifique que \(\cos \gamma = \sin\theta \sin\theta' \cos(\varphi - \varphi') + \cos\theta \cos\theta'\) pelo teorema da adição dos harmônicos esféricos.
        \item Utilize o teorema da adição dos harmônicos esféricos para escrever \(\norm{\D}^{-1}\) em termos dos ângulos \(\theta,\theta', \varphi\) e \(\varphi'\). Obtenha uma forma alternativa para a expansão multipolar de \(\phi(\vetor{\x})\).
    \end{enumerate}
\end{exercício}
\begin{proof}[Resolução]
\end{proof}
