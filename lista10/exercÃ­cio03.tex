\begin{lemma}{Relações de Fresnel na incidência normal}{incidência_normal}
    As relações para a incidência normal de um meio \(A\) para um meio \(B\) são dados por
    \begin{equation*}
        \tilde{E}^0_T = \frac{2}{1 + \beta} \tilde{E}^0_I \quad\text{e}\quad
        \tilde{E}^0_R = \frac{1 - \beta}{1 + \beta}\tilde{E}^0_I,
    \end{equation*}
    onde \(\beta = \frac{\mu_A v_A}{\mu_B v_B}\).
\end{lemma}
\begin{proof}
    Para uma onda incidente segundo a direção de propagação \(\vetor{e}_3\),
    \begin{equation*}
        \vetor{\tilde{E}}_I = \tilde{E}^0_I e^{i(k_Iz - \omega t)} \vetor{e}_1,
    \end{equation*}
    temos as ondas refletida e transmitida dadas por
    \begin{equation*}
        \vetor{\tilde{E}}_R = \tilde{E}^0_R e^{i(k_Rz - \omega t)} \vetor{e}_1\quad\text{e}\quad
        \vetor{\tilde{E}}_T = \tilde{E}^0_T e^{i(k_Tz - \omega t)} \vetor{e}_1.
    \end{equation*}
    Das condições de contorno temos
    \begin{equation*}
        \tilde{E}^0_I + \tilde{E}^0_R = \tilde{E}^0_T\quad\text{e}\quad\frac{1}{v_A\mu_A} (\tilde{E}^0_I - \tilde{E}^0_R) = \frac{1}{v_B\mu_B} \tilde{E}^0_T,
    \end{equation*}
    isto é,
    \begin{equation*}
        \tilde{E}^0_T - \tilde{E}^0_R = \tilde{E}^0_I\quad\text{e}\quad \beta \tilde{E}^0_T + \tilde{E}^0_R = \tilde{E}^0_I,
    \end{equation*}
    donde segue que
    \begin{equation*}
        \tilde{E}^0_T = \frac{2}{1 + \beta} \tilde{E}^0_I\quad\text{e}\quad \tilde{E}^0_R = \frac{1 - \beta}{1 + \beta} \tilde{E}^0_I,
    \end{equation*}
    como desejado.
\end{proof}
\begin{exercício}{Transmissão em cascata}{exercício03}
    Uma luz monocromática de frequência \(\omega\) passa de uma meio \(1\) para um meio \(3\), atravessando um meio \(2\) de espessura \(d\). Assumindo que os três meios são lineares, mostre que o coeficiente de transmissão para \emph{incidência normal} é dado por
    \begin{equation*}
        T^{-1} = \frac{1}{4n_1n_3} \left[(n_1 + n_3)^2 + \frac{(n_1^2 - n_2^2)(n_3^2 - n_2^2)}{n_2^2}\sin^2\left(\frac{n_2\omega d}{c}\right)\right].
    \end{equation*}
\end{exercício}
\begin{proof}[Resolução]
    Escrevamos as ondas no meio 1 como
    \begin{equation*}
        \vetor{\tilde{E}}^1_I = \tilde{E}^1_I e^{i(k_1z - \omega t)} \vetor{e}_1,\quad\text{e}\quad
        \vetor{\tilde{E}}^1_R = \tilde{E}^1_R e^{i(-k_1z - \omega t)} \vetor{e}_1,
    \end{equation*}
    nos meios 2 como
    \begin{equation*}
        \vetor{\tilde{E}}^2_I = \tilde{E}^2_I e^{i(k_2z - \omega t)} \vetor{e}_1,\quad\text{e}\quad
        \vetor{\tilde{E}}^2_R = \tilde{E}^2_R e^{i(-k_2z - \omega t)} \vetor{e}_1
    \end{equation*}
    e a onda transmitida no meio 3 como
    \begin{equation*}
        \vetor{\tilde{E}}_T = \tilde{E}^3_T e^{i(k_3z - \omega t)} \vetor{e}_1.
    \end{equation*}
    Como mostrado no \cref{lem:incidência_normal}, para a interface em \(z = d\) temos
    \begin{equation*}
        \tilde{E}_R^2 e^{-i k_2 d} = \frac{1 - \beta_{23}}{1 + \beta_{23}} \tilde{E}_I^2 e^{ik_2 d}
        \quad\text{e}\quad
        \tilde{E}_T^3 e^{ik_3 d} = \frac{2}{1 + \beta_{23}} \tilde{E}_I^2 e^{ik_2 d},
    \end{equation*}
    onde \(\beta_{23} = \frac{\mu_2 v_2}{\mu_3 v_3}\), isto é,
    \begin{equation*}
        \tilde{E}_I^2  = \frac{1 + \beta_{23}}{2} e^{i(k_3 - k_2) d} \tilde{E}_T^3
        \quad\text{e}\quad
        \tilde{E}_R^2  = \frac{1 - \beta_{23}}{2} e^{i(k_3 + k_2) d} \tilde{E}_T^3
    \end{equation*}
    Das condições de contorno para a interface em \(z = 0\) segue que
    \begin{equation*}
        \tilde{E}^1_I + \tilde{E}^1_R = \tilde{E}^2_I + \tilde{E}^2_R\quad\text{e}\quad \tilde{E}^1_I - \tilde{E}^1_R = \beta_{12} \left(\tilde{E}^2_I - \tilde{E}^2_R\right),
    \end{equation*}
    onde \(\beta_{12}=\frac{\mu_1v_1}{\mu_2 v_2}\). Com isso, obtemos
    \begin{align*}
        \tilde{E}_I^1 &= \frac{1 + \beta_{12}}{2} \tilde{E}_I^2 + \frac{1 - \beta_{12}}{2} \tilde{E}_R^2 \\
                      &= \frac12\left[\frac{(1 + \beta_{12})(1 + \beta_{23})}{2} e^{-i k_2 d} + \frac{(1 - \beta_{12})(1 - \beta_{23})}{2}e^{ik_2d}\right]e^{ik_3 d}\tilde{E}_T^3\\
                      &= \frac12\left[\frac{1 + \beta_{12} + \beta_{23} + \beta_{12}\beta_{23}}{2}e^{-ik_2 d} + \frac{1 - \beta_{12} - \beta_{23} + \beta_{12}\beta_{23}}{2}e^{ik_2d}\right]e^{ik_3 d}\tilde{E}_T^3\\
                      &= \frac12\left[\left(1 + \beta_{12} \beta_{23}\right)\frac{e^{ik_2 d} + e^{-ik_2d}}{2} - i\left(\beta_{12} + \beta_{23}\right)\frac{e^{ik_2d} - e^{-ik_2d}}{2i}\right]e^{ik_3 d}\tilde{E}_T^3\\
                      &= \frac12\left[\left(1 + \beta_{12} \beta_{23}\right)\cos(k_2 d) - i\left(\beta_{12} + \beta_{23}\right)\sin(k_2 d)\right]e^{ik_3 d}\tilde{E}_T^3,
    \end{align*}
    e temos
    \begin{align*}
        \abs*{\frac{\tilde{E}_I^1}{\tilde{E}_T^3}}
        &= \frac14 \left[(1 + \beta_{12} \beta_{23})^2 \cos^2(k_2 d) + (\beta_{12} + \beta_{23})^2 \sin^2(k_2d)\right]\\
        &= \frac14 \left\{(1 + \beta_{12} \beta_{23})^2 + \left[\left(\beta_{12} + \beta_{23}\right)^2 - (1 + \beta_{12} \beta_{23})^2\right]\sin^2(k_2 d)\right\}\\
        &= \frac14 \left[(1 + \beta_{12} \beta_{23})^2 + \left(\beta_{12}^2 + \beta_{23}^2 - 1 - \beta_{12}^2 \beta_{23}^2\right)\sin^2(k_2d)\right]\\
        &= \frac14 \left[(1 + \beta_{12} \beta_{23})^2 - (1 - \beta_{12}^2)(1 - \beta_{23}^2)\sin^2(k_2d)\right],
    \end{align*}
    logo, o coeficiente de transmissão é dado por
    \begin{equation*}
        T^{-1} = \frac{v_1 \epsilon_1 (E_I^1)^2}{v_3 \epsilon_3 (E_T^3)^2} = \frac{\mu_1v_1 \mu_3\epsilon_1}{\mu_3v_3 \mu_1\epsilon_3}\abs*{\frac{\tilde{E}_I^1}{\tilde{E}_T^3}} = \frac{\mu_3 \epsilon_1 \beta_{12} \beta_{23}}{4 \mu_1 \epsilon_3} \left[(1 + \beta_{12} \beta_{23})^2 - (1 - \beta_{12}^2)(1 - \beta_{23}^2)\sin^2(k_2d)\right].
    \end{equation*}
    Se \(\mu_1 \approx \mu_2 \approx \mu_3 \approx \mu_0\), temos \(\beta_{12} = \frac{n_2}{n_1}\) e \(\beta_{23} = \frac{n_3}{n_2}\), portanto
    \begin{align*}
        T^{-1} &\approx \frac{\mu_1 \epsilon_1 n_3}{4 \mu_3 \epsilon_3 n_1} \left[\left(1 + \frac{n_3}{n_1}\right)^2 - \left(1 - \frac{n_2^2}{n_1^2}\right)\left(1 - \frac{n_3^2}{n_2^2}\right)\sin^2\left(\frac{n_2 \omega d}{c}\right)\right]\\
               &= \frac{v_3^2 n_3}{4v_1^2 n_1} \left[\frac{1}{n_1^2}(n_1 + n_3)^2 + \frac{1}{n_1^2n_2^2} (n_1^2 - n_2^2)(n_3^2 - n_2^2)\sin^2\left(\frac{n_2 \omega d}{c}\right)\right]\\
               &= \frac{1}{4 n_1 n_3} \left[(n_1 + n_3)^2 + \frac{(n_1^2 - n_2^2) (n_3^2 - n_2^2)}{n_2^2} \sin^2\left(\frac{n_2 \omega d}{c}\right)\right]
    \end{align*}
    como desejado.
\end{proof}
