\begin{exercício}{Coeficiente de transmissão de incidência normal}{exercício03}
    Uma luz monocromática de frequência \(\omega\) passa de uma meio \(1\) para um meio \(3\), atravessando um meio \(2\) de espessura \(d\). Assumindo que os três meios são lineares, mostre que o coeficiente de transmissão para \emph{incidência normal} é dado por
    \begin{equation*}
        T^{-1} = \frac{1}{4n_1n_3} \left[(n_1 + n_3)^2 + \frac{(n_1^2 - n_2^2)(n_3^2 - n_2^2)}{n_2^2}\sin^2\left(\frac{n_2\omega d}{c}\right)\right].
    \end{equation*}
    \todo[Tomar \(\mu_1 \approx \mu\)].
\end{exercício}
