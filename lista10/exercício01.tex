\begin{exercício}{Média temporal em notação complexa}{exercício01}
    Ao estudar as densidades de energia e de momento linear associadas a uma onda eletromagnética plana e monocromática, foi conveniente calcular \emph{médias} dessas quantidades sobre um ciclo de oscilação \(\frac{2\pi}{\omega}\). Em notação complexa, há uma maneira simples de calcular médias de produtos de ondas senoidais. Sejam \(f(\vetor{\x}, t) = A \cos(\vetor{k}\cdot \vetor{\x} - \omega t + \delta_f)\) e \(g(\vetor{\x}, t) = B \cos(\vetor{k}\cdot \vetor{\x} - \omega t + \delta_g)\) duas ondas escalares com o mesmo vetor de onda \(\vetor{k}\) e frequência \(\omega\). Denotemos também por \(\tilde{f}(\vetor{\x}, t) = \tilde{A} e^{i(\vetor{k}\cdot\vetor{\x} - \omega t)}\) e \(\tilde{g}(\vetor{\x}, t) = \tilde{B} e^{i(\vetor{k}\cdot \vetor{\x} - \omega t)}\) as versões complexas de \(f\) e \(g\). Mostre que
    \begin{equation*}
        \mean{fg} = \frac12 \Re{(\tilde{f} \conj{\tilde{g}})},
    \end{equation*}
    e conclua, no contexto de ondas planas, que
    \begin{equation*}
        \mean{u} = \frac14 \Re{\left(\epsilon_0 \vetor{\tilde{E}} \cdot \conj{\vetor{\tilde{E}}} + \frac1{\mu_0} \vetor{\tilde{B}}\cdot \conj{\vetor{\tilde{B}}}\right)}
        \quad\text{e}\quad
        \mean{\vetor{S}} = \frac1{2\mu_0} \Re(\vetor{\tilde{E}}\times \conj{\vetor{\tilde{B}}}).
    \end{equation*}
\end{exercício}
\begin{proof}[Resolução]
    Notemos que
    \begin{align*}
        f g = \Re(\tilde{f})\Re(\tilde{g}) &= \frac14(\tilde{f} + \conj{\tilde{f}})(\tilde{g} + \conj{\tilde{g}})\\
                                           &= \frac14(\tilde{f} \tilde{g} + \conj{\tilde{f}}\tilde{g} + \tilde{f}\conj{\tilde{g}} + \conj{\tilde{f}}\conj{\tilde{g}})\\
                                           &= \frac14 ( \tilde{f} \tilde{g} + \conj{\tilde{f}}\conj{\tilde{g}}) + \frac14 (\conj{\tilde{f}}\tilde{g} + \tilde{f} \conj{\tilde{g}}).
    \end{align*}
    Fatoremos a dependência temporal de forma que \(\tilde{F}(\vetor{\x}) = \tilde{f}(\vetor{\x}, t)e^{i\omega t}\) e analogamente para \(\tilde{g}\), então
    \begin{equation*}
        \conj{\tilde{f}}(\vetor{\x}, t)\tilde{g}(\vetor{\x}, t) + \tilde{f}(\vetor{\x}, t) \conj{\tilde{g}}(\vetor{\x}, t) = \conj{\tilde{F}}(\vetor{\x}) \tilde{G}(\vetor{\x}) + \tilde{F}(\vetor{\x})\conj{\tilde{G}}(\vetor{\x})
    \end{equation*}
    independe do tempo e
    \begin{align*}
        \tilde{f}(\vetor{\x}, t) \tilde{g}(\vetor{\x}, t) + \conj{\tilde{f}}(\vetor{\x}, t)\conj{\tilde{g}}(\vetor{\x}, t)
        &= \tilde{F}(\vetor{\x}) \tilde{G}(\vetor{\x}) e^{-2i\omega t} + \conj{\tilde{F}}(\vetor{\x}) \conj{\tilde{G}}(\vetor{\x})e^{2i\omega t}\\
        &= 2\Re\left(\tilde{F}(\vetor{\x})\tilde{G}(\vetor{\x})e^{-2i\omega t}\right)\\
        &= 2\Re\left(\tilde{F}(\vetor{\x})\tilde{G}(\vetor{\x})\right)\cos(2\omega t) + 2\Im\left(\tilde{F}(\vetor{\x})\tilde{G}(\vetor{\x})\right)\sin(2\omega t).
    \end{align*}
    Isto é, temos \(\mean{\tilde{f} \tilde{g} + \conj{\tilde{f}}\conj{\tilde{g}}} = 0\), já que \(\mean{\cos(2\omega t)} = \mean{\sin(2\omega t)} = 0\), donde segue que
    \begin{equation*}
        \mean{f g} = \frac14 (\conj{\tilde{f}}\tilde{g} + \tilde{f}\conj{\tilde{g}}) = \frac12 \Re(\tilde{f}\conj{\tilde{g}}),
    \end{equation*}
    como desejado.
\end{proof}
