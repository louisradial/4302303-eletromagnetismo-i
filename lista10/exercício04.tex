\begin{exercício}{Incidência oblíqua em interface de meios materiais}{exercício04}
    Em aula, resolvemos o problema de incidência \emph{oblíqua} sobre uma interface que divide dois meios materiais lineares 1 e 2 para o caso onde a polarização é \emph{paralela} ao plano de incidência. Analise agora o caso onde a polarização é \emph{perpendicular} ao plano de incidência. Obtenha as relações de Fresnel para as amplitudes \(\tilde{E}_R^0\) e \(\tilde{E}_T^0\). Esboce gráficos das quantidades \(\tilde{E}_R^0/\tilde{E}_I^0\) e \(\tilde{E}_T^0/\tilde{E}_I^0\) em função do ângulo  de incidência \(\theta_I\) para o caso particular \(\beta = n_2/n_1 = \frac32\). Mostre que a menos que \(n_2 = n_1\) e \(\mu_2 = \mu_1\) (o que significa que os meios são indistinguíveis), não existe um ângulo de Brewster para o caso da polarização perpendicular. Por fim, calcule os coeficientes de transmissão e reflexão e verifique que \(T + R = 1\).
\end{exercício}
\begin{proof}[Resolução]

\end{proof}
