\begin{exercício}{Polarização das ondas refletida e transmitida na incidência normal}{exercício02}
    Em aula resolvemos o problema da incidência normal de uma onda plana monocromática sobre uma interface entre dois materiais lineares. Orientamos o sistema de coordenadas de forma que o eixo \(z\) fosse perpendicular ao plano da interface (esta, por sua vez, se encontra em \(z = 0\)) e fosse também a direção de propagação da onda incidente, tal que \(\vetor{k}_I = k_I \vetor{e}_3\). Além disso, os eixos \(x\) e \(y\) foram orientados de forma que a polarização do campo elétrico da onda incidente fosse paralela ao eixo \(x\), de forma que escrevemos
    \begin{equation*}
        \vetor{\tilde{E}}_I(\vetor{\x},t) = \tilde{E}^0_I e^{i(k_i z - \omega t)}\vetor{e}_1\quad\text{e}\quad
        \vetor{\tilde{B}}_I(\vetor{\x}, t) = \frac{1}{\omega} \vetor{k}_I \times \vetor{\tilde{E}}_I(\vetor{\x},t).
    \end{equation*}
    Porém, ao escrever expressões para os campos refletido (no meio 1) e transmitido (no meio 2), assumimos que a polarização dos campos elétricos \(\vetor{\tilde{E}}_R\) e \(\vetor{\tilde{E}}_T\) eram também paralelas à \(\vetor{e}_1\). É claro que, sendo as condas eletromagnéticas transversais, as polarizações dessas duas ondas não podem ter componentes \(z\). Mas não é imediatamente claro porque não há, por exemplo, componentes \(y\). Use as condições de contorno para provar que esse é \emph{necessariamente} o caso. Isto é, mostre que se os campos incidentes são dados pelas expressões acima, então só haverá componentes \(x\) dos campos \(\vetor{\tilde{E}}_R\) e \(\vetor{\tilde{E}}_T\).
\end{exercício}
