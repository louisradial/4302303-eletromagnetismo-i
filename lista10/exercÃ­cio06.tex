\begin{exercício}{Onda eletromagnética plana monocromática em um condutor}{exercício06}
    Calcule a média temporal (sobre um período de oscilação) da densidade de energia de uma onda eletromagnética plana monocromática de frequência \(\omega\) em um meio condutor de permissividade elétrica \(\epsilon\), permeabilidade magnética \(\mu\) e condutividade \(\sigma\). Mostre que a contribuição magnética é sempre dominante em relação à contribuição elétrica. Em seguida, mostre que a intensidade pode ser escrita como
    \begin{equation*}
        I = \frac{\kappa_+}{2\mu \omega} E_0^2 e^{-2 \kappa_-z},
    \end{equation*}
    em que
    \begin{equation*}
        \kappa_\pm = \omega \sqrt{\frac{\epsilon \mu}{2}}\left[\sqrt{1 + \left(\frac{\sigma}{\epsilon \omega}\right)^2} \pm 1\right]^{\frac12}.
    \end{equation*}
\end{exercício}
