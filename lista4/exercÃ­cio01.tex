\begin{exercício}{Teorema do valor médio da eletrostática}{exercício1}
    Mostre que em um aberto \(\Sigma\subset \mathbb{R}^3\) sem cargas, para todo \(\vetor{\x} \in \Sigma\) vale
    \begin{equation*}
        \phi(\vetor{\x}) = \frac{\oint_{\partial \Omega}\dln2{\x'} \phi(\vetor{\x'})}{\oint_{\partial \Omega} \dln2{\x'}},
    \end{equation*}
    onde \(\Omega\) é qualquer região esférica centrada em \(\vetor{\x}\) e contida em \(\Sigma\).
\end{exercício}
\begin{proof}[Resolução]

\end{proof}
