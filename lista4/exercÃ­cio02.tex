\begin{exercício}{Cubo de paredes condutoras}{exercício2}
    Considere um cubo de paredes condutoras em que cada uma das seis faces é mantida a um potencial \(\phi_i\), com \(i \in \set{1,\dots, 6}\). Considerando que não há cargas em seu interior obtenha o valor do potencial no centro do cubo, apenas por argumentos baseados em simetria, no princípio da superposição e na unicidade de soluções.
\end{exercício}
\begin{proof}[Resolução]
    Consideremos problemas em que uma das seis faces é mantida a um potencial \(\tilde{\phi}\) e as demais são aterradas, e mostremos que o potencial \(\tilde{\phi}_c\) no centro do cubo é \(\frac16\tilde{\phi}\). Se trocássemos a face que não é aterrada, o potencial no centro do cubo não deve ser diferente, visto que podemos orientar o cubo de forma que a face mantida ao potencial \(\tilde{\phi}\) é a original. Isto é, o potencial no centro do cubo é o mesmo para qualquer problema deste tipo. Consideremos a superposição destes problemas, que resulta no problema em que todas as faces do cubo são mantidas a este potencial, portanto o potencial no centro é \(6\tilde{\phi}_c\), pelo princípio da superposição. Notemos que o o potencial constante \(\tilde{\phi}\) é solução para a equação de Laplace com esta condição de contorno, portanto \(\tilde{\phi}_c = \frac16 \tilde{\phi}\) pela unicidade de soluções.

    Assim, para o problema em que uma face é mantida ao potencial \(\phi_i\), o potencial no centro é \(\frac{1}{6}\phi_i\). Como o problema em que cada face é mantida a um potencial distinto pode ser visto como a superposição de problemas de que as demais faces são aterradas, o potencial no centro do cubo será dado pela soma do potencial no centro do cubo de cada um destes problemas, isto é, o potencial no centro do cubo é a média simples dos potenciais das faces, \(\frac16 \sum_{i=1}^6 \phi_i\).
\end{proof}
