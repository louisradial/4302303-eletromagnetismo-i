\begin{exercício}{Potencial de uma esfera com condição de contorno com simetria azimutal}{exercício6}
    O potencial na superfície de uma casca esférica de raio \(R\) é dado por
    \begin{equation*}
        \phi_0(\theta) = k \cos(3\theta),
    \end{equation*}
    onde \(k\) é uma constante não nula. Encontre o potencial dentro e fora da casca, bem como a densidade superficial de carga \(\sigma(\theta)\) na superfície.
\end{exercício}
\begin{proof}[Resolução]
    Notemos que
    \begin{align*}
        \cos(3\theta) &= \cos(2\theta)\cos\theta - \sin(2\theta) \sin\theta\\&= (2\cos^2\theta - 1)\cos\theta - 2(1 - \cos^2\theta)\cos\theta \\&= 4\cos^3\theta - 3\cos\theta
    \end{align*}
    então por inspeção temos que
    \begin{equation*}
        \phi_0(\theta) = k\left[\frac85P_3(\cos\theta) - \frac35 P_1(\cos\theta)\right].
    \end{equation*}
    Desse modo, temos
    \begin{equation*}
        \frac{2 \ell + 1}{2}C_\ell = \frac{2\ell + 1}{2}\int_0^\pi \sin\psi \dli{\psi} \phi_0(\psi) P_\ell(\cos\psi) = \left(\frac{8}{5}\delta_{\ell 3} - \frac{3}{5}\delta_{\ell 1}\right)k
    \end{equation*}
    portanto o potencial elétrico em todo espaço é
    \begin{equation*}
        \phi(r, \theta) = \begin{cases}
            k\left[\frac{4(5 \cos^3\theta - 3 \cos\theta)}{5}\left(\frac{r}{R}\right)^3 - \frac{3\cos\theta}{5}\left(\frac{r}{R}\right)\right], &\text{se } r < R\\
            \phi_0(\theta), &\text{se } r = R\\
            k\left[\frac{4(5 \cos^3\theta - 3 \cos\theta)}{5}\left(\frac{R}{r}\right)^4 - \frac{3\cos\theta}{5}\left(\frac{R}{r}\right)^2\right], &\text{se } r > R
        \end{cases}
    \end{equation*}
    e a densidade superficial de carga na superfície é
    \begin{align*}
    \sigma &= \frac{\epsilon_0}{2R}\sum_{\ell = 0}^\infty (2\ell + 1)^2C_\ell P_\ell(\cos\theta)\\
           &= \frac{k\epsilon_0}{R} \sum_{\ell = 0}^\infty (2 \ell + 1) \left(\frac{8}{5}\delta_{\ell 3} - \frac{3}{5} \delta_{\ell 1}\right) P_\ell(\cos\theta)\\
           &= \frac{k \epsilon_0}{5R}\left[56 P_3(\cos \theta) - 9 P_1(\cos\theta)\right]\\
           &= \frac{k \epsilon_0}{5 R}\left[140 \cos^3\theta - 93\cos\theta\right],
    \end{align*}
    pelo \cref{ex:exercício5}.
\end{proof}
