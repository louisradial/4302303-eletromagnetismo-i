\begin{exercício}{Cubo de paredes condutoras}{exercício2}
    Considere um cubo de paredes condutoras em que cada uma das seis faces é mantida a um potencial \(\phi_i\), com \(i \in \set{1,\dots, 6}\). Considerando que não há cargas em seu interior obtenha o valor do potencial no centro do cubo, apenas por argumentos baseados em simetria, no princípio da superposição e na unicidade de soluções.
\end{exercício}
\begin{proof}[Resolução]

\end{proof}
