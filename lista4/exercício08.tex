\begin{exercício}{Solução da equação de Laplace com simetria de translação axial}{exercício8}
    Resolva a equação de Laplace pela separação de variáveis em coordenadas cilíndricas assumindo que não há dependência em \(z,\) encontrando uma expressão para a solução geral.
\end{exercício}
\begin{proof}[Resolução]
    Se o potencial é invariante por translações axiais, temos \(\phi = \phi(s, \theta)\). Buscando uma solução produto, \(\phi(s, \theta) = R(s)\Xi(\theta)\), temos
    \begin{equation*}
        0 = \nabla^2 \phi = \frac1s \diffp*{\left(s \diff{R}{s} \Xi(\theta)\right)}{s} + \frac{1}{s^2}R(s)\diff[2]{\Xi}{\theta} = \diff[2]{R}{s}\Xi(\theta) + \frac{1}{s} \diff{R}{s}\Xi(\theta) + \frac{1}{s^2} R(s) \diff[2]{\Xi}{\theta},
    \end{equation*}
    portanto
    \begin{equation*}
        \frac{s^2}{R(s)} \diff[2]{R}{s} + \frac{s}{R(s)} \diff{R}{s} = - \frac{1}{\Xi(\theta)} \diff[2]{\Xi}{\theta}.
    \end{equation*}
    Como o lado esquerdo é função só de \(s\) e o lado esquerdo é função só de \(\theta\), ambos devem ser constantes, isto é, existe \(\lambda\in \mathbb{R}\) constante tal que
    \begin{equation*}
        s^2 \diff[2]{R}{s} + s\diff{R}{s} - \lambda R(s) = 0 \quad\text{e}\quad \diff[2]{\Xi}{\theta} + \lambda \Xi(\theta) = 0.
    \end{equation*}
    Como \(\Xi\) deve ser periódica, devemos ter \(\lambda \geq 0\), caso contrário haveria apenas a solução trivial. Notemos que para \(\lambda = 0\), temos a solução constante \(\Xi_0(\theta) = a_0\) e a solução para \(R\) é da forma \(R_0(s) = c_0 \ln s + d_0\). Para \(\lambda > 0\) podemos escrever \(\lambda = m^2\), o período de \(\Xi\) deve ser \(2\pi\), donde concluímos que
    \begin{equation*}
        \Xi_m(\theta) = a_m\cos(m\theta) + b_m \sin(m\theta),
    \end{equation*}
    com \(m \in \mathbb{N}\). Notemos que a equação para \(R\) é uma equação diferencial de Euler, portanto tentamos uma solução do tipo \(R(s) = s^\nu\) e obtemos
    \begin{equation*}
        \left[\nu(\nu - 1) + \nu - m\right]s^\nu = 0 \implies \nu = \pm m,
    \end{equation*}
    isto é,
    \begin{equation*}
        R(s) = c_ms^m + d_m s^{-m},
    \end{equation*}
    uma vez que \(\set{s^m, s^{-m}}\) é um conjunto linearmente independente de soluções. Concluímos que
    \begin{equation*}
        \phi(s, \theta) = \alpha + \beta \ln(s) + \sum_{m \in \mathbb{N}} \left[a_m \cos(m\theta) + b_m\sin(m\theta)\right](c_m s^m + d_m s^{-m})
    \end{equation*}
    é a solução geral para a equação de Laplace com simetria de translação axial.
\end{proof}
