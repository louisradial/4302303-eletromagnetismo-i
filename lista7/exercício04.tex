\begin{exercício}{Força entre placas paralelas carregadas se movendo com velocidade constante}{exercício4}
    Duas placas paralelas muito grandes estão carregadas com densidades de carga superficiais uniformes \(\sigma\) (placa superior) e \(-\sigma\) (placa inferior). Você pode considerar que as placas são paralelas ao plano \(xy\). Além disso, as duas placas estão se movendo na mesma direção a uma velocidade constante \(\vetor{v} = v\vetor{e}_x\).
    \begin{enumerate}[label=(\alph*)]
        \item Encontre o campo magnético na região entre as placas e na região externa.
        \item Obtenha uma expressão para a força magnética por unidade de área na placa superior.
        \item Com que velocidade \(v\) a força magnética equilibraria a força elétrica?
    \end{enumerate}
\end{exercício}
\begin{proof}[Resolução]

\end{proof}
