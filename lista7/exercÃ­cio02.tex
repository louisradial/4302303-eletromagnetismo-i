\begin{exercício}{Campo magnético no eixo de simetria de um disco giratório}{exercício2}
    Encontre o campo magnético em um ponto sobre o eixo \(z\) acima de um disco giratório de raio \(R\) carregado com densidade de carga superficial de carga \(\sigma\). Considere que o plano do disco coincide com o plano \(xy\) e que o mesmo gira em torno do eixo \(z\) com velocidade angular \(\omega\). Caso o problema tratasse de encontrar o campo magnético sobre o eixo \(z\) gerado por uma esfera maciça giratória de raio \(R\) (para pontos com \(z > R\)), carregada com densidade de carga volumétrica \(\rho\), você conseguiria propor uma estratégia que se aproveita da resposta anterior? Não é necessário resolver a conta, apenas deixe claro qual seria a estratégia, como os parâmetros \(\sigma\) e \(\rho\) se relacionam e como seria a integral.
\end{exercício}
\begin{proof}[Resolução]

\end{proof}
