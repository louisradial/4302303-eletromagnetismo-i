\begin{exercício}{Força magnética devido a um circuito fechado em um outro circuito fechado}{exercício3}
    Dois fios que formam circuitos fechados descritos pelas curvas \(\Gamma_1\) e \(\Gamma_2\) são percorridos, respectivamente, pelas correntes estacionárias \(I_1\) e \(I_2\).
    Mostre que a força magnética sobre todo o circuito 1, devido a sua interação com o circuito 2, pode ser escrita como
    \begin{equation*}
        \vetor{F}_{1(2)} =-\frac{\mu_0 I_1 I_2}{4\pi} \oint_{\Gamma_1}\oint_{\Gamma_2}{\dl{\ell_1}\cdot \dli{\ell_2} \frac{\vetor{x}_1 - \vetor{x}_2}{\norm{\vetor{\x}_1 - \vetor{\x}_2}^3}}
    \end{equation*}
\end{exercício}
\begin{proof}[Resolução]

\end{proof}
