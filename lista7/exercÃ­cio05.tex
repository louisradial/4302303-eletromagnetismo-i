\begin{exercício}{Condutor infinitamente longo com cavidade}{exercício5}
    A figura abaixo representa a seção transversal de um cabo infinitamente longo. A região hachurada é feita de um material condutor e conduz uma corrente elétrica de densidade constante do tipo \(\vetor{J} = J\vetor{e}_z\), enquanto que a região não hachurada (de raio \(b\)) é uma cavidade que não conduz corrente. Encontre o campo magnético em todos os pontos do eixo \(x\).
    \begin{center}
        \begin{tikzpicture}[scale=0.5, every node/.style={scale=0.7}]
            \fill[Overlay0!95] (0,0) circle (4);
            \fill[Sky!5] (2,0) circle (1);
            \draw[-stealth] (-5,0) -- (5,0) node[anchor=north] {\(x\)};
            \draw[-stealth] (0,-5) -- (0,5) node[anchor=west] {\(y\)};
            \draw (0,0) -- ({4*cos(135)},{4*sin(135)}) node[midway, above] {\(a\)};
            \draw (2,0) -- ({2 + 1*cos(135)},{1*sin(135)}) node[midway, above] {\(b\)};
            \draw[<->] (0,-4.5) -- (2,-4.5) node[midway, anchor=north] {\(d\)};
            \draw[dotted] (2,0) -- (2,-4.5);
        \end{tikzpicture}
    \end{center}
\end{exercício}
\begin{proof}[Resolução]
    Consideremos um cabo infinitamente longo de raio \(R\) e densidade de corrente constante \(\vetor{\tilde{J}} = \tilde{J}\vetor{e}_z\) cujo eixo de simetria passa pelo ponto \(\vetor{O}\). O potencial vetor no gauge de Coulomb é
    \begin{align*}
        \vetor{\tilde{A}}(\vetor{O} + s \vetor{e}_s + z \vetor{e}_z)
        &= \frac{\mu_0}{4\pi} \int_0^R \dli{s'} \int_{\mathbb{R}}\dli{z'} \int_0^{2\pi}s' \dli{\varphi'} \frac{\tilde{J}\vetor{e}_z}{\sqrt{(z - z')^2 + s^2 + s'^2 - 2ss'\cos(\varphi' - \varphi)}}\\
        &= \frac{\mu_0\tilde{J}}{4\pi} \int_0^R \dli{s'} \int_{\mathbb{R}}\dli{\tilde{z}} \int_0^{2\pi} \dli{\tilde{\varphi}} \left(\tilde{z}^2 + s^2 + s'^2 - 2ss'\cos\tilde\varphi\right)^{-\frac12}\vetor{e}_z,
    \end{align*}
    portanto vemos que \(\vetor{\tilde{A}}(\vetor{O} + s\vetor{e}_s + z\vetor{e}_z) = \tilde{A}(s)\vetor{e}_z\), com
    \begin{equation*}
        \lim_{s \to \infty} \tilde{A}(s) = 0
        \quad\text{e}\quad
        \lim_{s \to \infty} \diffp{\tilde{A}}{s} = 0.
    \end{equation*}
    Com isso, sabemos que \(\vetor{\tilde{B}}(\vetor{O} + s\vetor{e}_s + z\vetor{e}_z) = \tilde{B}(s)\vetor{e}_{\varphi},\) com \(B(s) \to 0\) conforme \(s \to \infty\). Consideremos um disco \(\Omega\) de raio \(r\) centrado em \(\vetor{O}\) e transversal ao cabo, e um caminho fechado \(\Gamma\) positivamente orientado que percorre a circunferência de raio \(r\) determinada por \(\Omega\). A corrente que atravessa \(\Omega\) é dada por
    \begin{equation*}
        \int_{\Omega}\dln2{\x'} \inner{\vetor{e}_z}{\vetor{\tilde{J}}(\vetor{\x'})} = \begin{cases}
            \tilde{J}\pi R^2,&\text{se }r \geq R\\
            \tilde{J}\pi r^2,&\text{se }r < R
        \end{cases}
    \end{equation*}
    e a circulação de \(\vetor{B}\) por \(\Gamma\) é
    \begin{equation*}
        \oint_{\Gamma} \dli{\vetor{\ell}}\cdot \vetor{\tilde{B}} = 2\pi r \tilde{B}(r),
    \end{equation*}
    portanto pela lei de Ampère, temos
    \begin{equation*}
        \oint_{\Gamma} \dli{\vetor{\ell}}\cdot \vetor{\tilde{B}} = \mu_0\int_{\Omega}\dln2{\x'} \inner{\vetor{e}_z}{\vetor{\tilde{J}}(\vetor{\x'})}  \implies \tilde{B}(s) = \begin{cases}
            \frac{\mu_0\tilde{J}R^2}{2r},&\text{se }r > R\\
            \frac{\mu_0\tilde{J}r}{2},&\text{se }r < R.
        \end{cases}
    \end{equation*}
    Isto é, o campo devido ao cabo é dado por
    \begin{equation*}
        \vetor{\tilde{B}}(\vetor{O} + x\vetor{e}_x + y \vetor{e}_y + z\vetor{e}_z) = \begin{cases}
            \displaystyle\frac{\mu_0\tilde{J}R^2(- y\vetor{e}_x + x \vetor{e}_y)}{2 (x^2 + y^2)},&\text{se }x^2 + y^2 > R^2\\
            \displaystyle\frac{\mu_0\tilde{J}}{2}( - y\vetor{e}_x + x \vetor{e}_y),&\text{se }x^2 + y^2 < R^2.
        \end{cases}
    \end{equation*}

    Com isso, a configuração considerada pode ser entendida como a superposição de dois cabos infinitamente longos de raios \(a\) e \(b\), centrados na origem e na posição \(d\vetor{e}_x\), cujas densidades de corrente são \(\vetor{J}\) e \(-\vetor{J}\), e então
    \begin{equation*}
        \vetor{B}(\vetor{\x}) = \frac{\mu_0J}{2}
        \left[\begin{cases}
                \frac{a^2(- y\vetor{e}_x + x \vetor{e}_y)}{x^2 + y^2},&\text{se }x^2 + y^2 > a^2\\
                - y\vetor{e}_x + x \vetor{e}_y,&\text{se }x^2 + y^2 < a^2
            \end{cases}
        -
        \begin{cases}
                \frac{b^2(- y\vetor{e}_x + (x - d) \vetor{e}_y)}{(x-d)^2 + y^2},&\text{se }(x-d)^2 + y^2 > b^2\\
                - y\vetor{e}_x + (x-d) \vetor{e}_y,&\text{se }(x-d)^2 + y^2 < b^2
            \end{cases}
        \right]
    \end{equation*}
    é o campo magnético em todo o espaço, com \(\vetor{\x} = x\vetor{e}_x + y\vetor{e}_y + z\vetor{e}_z\). Em particular,
    \begin{align*}
        \vetor{B}(x\vetor{e}_x) &= \frac{\mu_0J}{2}
        \left[\begin{cases}
                \frac{a^2 }{x},&\text{se }\abs{x} > a\\
                x,&\text{se }\abs{x} < a
            \end{cases}
        -
        \begin{cases}
                \frac{b^2}{x-d},&\text{se }\abs{x-d}> b\\
                x-d,&\text{se }\abs{x-d}< b
            \end{cases}
        \right]\vetor{e}_y\\
                                &= \begin{cases}
                                    \frac{\mu_0Jd}{2}\vetor{e}_y,&\text{se }\abs{x - d} < b\\
                                    \frac{\mu_0J}{2}\left(x - \frac{b^2}{x - d}\right)\vetor{e}_y,&\text{se }\abs{x - d} > b\text{ e }\abs{x} < a\\
                                    \frac{\mu_0J}{2}\left(\frac{a^2}{x} - \frac{b^2}{x - d}\right)\vetor{e}_y,&\text{se }\abs{x} > a
                                \end{cases}
    \end{align*}
    é o campo no eixo \(x\).
\end{proof}
