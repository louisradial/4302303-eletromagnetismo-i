\begin{exercício}{Campo magnético de um fio longo com ângulo reto}{exercício1}
    Um fio muito longo carrega uma corrente estacionária \(I\) e faz uma curva de ângulo reto como abaixo. Calcule o campo magnético no ponto \(P\) indicado em cada uma das duas situações indicadas abaixo.
    \begin{center}
        \begin{tikzpicture}
            \node at (1,-2.5) {Configuração (a)};
            \draw[thick] (0,-2) -- (0,0);
            \draw[thick, -stealth] (0,-2) -- (0,-1);
            \draw[thick, -stealth] (0,0) -- (1,0) node[above] {\(I\)};
            \draw[thick] (0,0) -- (2,0);
            \filldraw[thick] (0, 1) circle(0.01) node[anchor=west] {\(P\)};
            \draw[dotted] (0,0) -- (0,1) node[midway,anchor=east] {\(d\)};
            \begin{scope}[xshift=7cm]
                \node at (1,-2.5) {Configuração (b)};
                \draw[thick] (0,-2) -- (0,0);
                \draw[thick, -stealth] (0,-2) -- (0,-1);
                \draw[thick, -stealth] (0,0) -- (1,0) node[above] {\(I\)};
                \draw[thick] (0,0) -- (2,0);
                \filldraw[thick] +(135:1) circle(0.01) node[anchor=south] {\(P\)};
                \draw[dotted] (0,0) -- +(135:1) node[midway,anchor=north] {\(d\)};
                \draw[dotted] (0,0) -- +(315:0.5);
                \draw[thick] (0.5,0) arc[start angle=360, end angle=315, radius=0.5]  node[anchor=west] {\(\frac{\pi}{4}\)};
            \end{scope}
        \end{tikzpicture}
    \end{center}
\end{exercício}
\begin{proof}[Resolução]

\end{proof}
