\begin{exercício}{Condutor infinitamente longo com cavidade}{exercício5}
    A figura abaixo representa a seção transversal de um cabo infinitamente longo. A região hachurada é feita de um material condutor e conduz uma corrente elétrica de densidade constante do tipo \(\vetor{J} = J\vetor{e}_z\), enquanto que a região não hachurada (de raio \(b\)) é uma cavidade que não conduz corrente. Encontre o campo magnético em todos os pontos do eixo \(x\).
    \begin{center}
        \begin{tikzpicture}[scale=0.5, every node/.style={scale=0.7}]
            \fill[Overlay0!95] (0,0) circle (4);
            \fill[Sky!5] (2,0) circle (1);
            \draw[-stealth] (-5,0) -- (5,0) node[anchor=north] {\(x\)};
            \draw[-stealth] (0,-5) -- (0,5) node[anchor=west] {\(y\)};
            \draw (0,0) -- ({4*cos(135)},{4*sin(135)}) node[midway, above] {\(a\)};
            \draw (2,0) -- ({2 + 1*cos(135)},{1*sin(135)}) node[midway, above] {\(b\)};
            \draw[<->] (0,-4.5) -- (2,-4.5) node[midway, anchor=north] {\(d\)};
            \draw[dotted] (2,0) -- (2,-4.5);
        \end{tikzpicture}
    \end{center}
\end{exercício}
\begin{proof}[Resolução]

\end{proof}
