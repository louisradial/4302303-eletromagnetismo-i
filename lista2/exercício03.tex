\begin{exercício}{Campo elétrico gerado por um disco em seu eixo de simetria}{exercício3}
    Encontre o campo elétrico a uma distância \(z\) acima do centro de um disco circular de raio \(R\), que tem uma densidade superficial de carga uniforme \(\sigma\). Cheque sua resposta no limite \(R \to \infty\) e obtenha assim o campo elétrico gerado por um plano infinito.
\end{exercício}
\begin{proof}[Resolução]
    Situemos o sistema de coordenadas com a origem no centro do disco e orientando o eixo axial \(\vetor{e}_z\) na direção perpendicular ao disco e consideremos o ponto \(P\) com vetor posição \(\vetor{\x} = z\vetor{e}_z\). Assim, o campo elétrico no ponto \(P\) é dado por
    \begin{align*}
        \vetor{E}(\vetor{\x}) &= \frac1{4\pi \epsilon_0} \int_0^R\dli{s} \int_0^{2\pi} s\dli{\theta} \sigma\frac{z\vetor{e}_z - s\left(\cos\theta \vetor{e}_x + \sin\theta \vetor{e}_y\right)}{\left(z^2 + s^2\right)^{\frac32}}\\
                              &= \frac{z \sigma\vetor{e}_z}{2\epsilon_0} \int_0^R s\dl{s} \left(z^2 + s^2\right)^{-\frac32}\\
                              &= \frac{z \sigma}{2 \epsilon_0}\left[\frac1{\abs{z}} - \frac1{\sqrt{z^2 + R^2}}\right]\vetor{e}_z\\
                              &= \frac{\sigma}{2 \epsilon_0}\left[\sgn{z} - \frac{z}{\sqrt{z^2 + R^2}}\right]\vetor{e}_z.
    \end{align*}
    No limite \(R \to \infty\), temos \(\vetor{E} = \frac{\sigma}{2 \epsilon_0} \sgn(z)\vetor{e}_z\), que é o campo elétrico gerado por um plano infinito uniformemente carregado.
\end{proof}
