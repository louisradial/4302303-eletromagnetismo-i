\begin{exercício}{Comparação entre energia eletrostática em um isolante e em um condutor}{exercício11}
    Compare a energia eletrostática de uma esfera maciça isolante uniformemente carregada e uma esfera condutora, ambas de carga total \(Q\) e raio \(R\). Use essa diferença para tentar argumentar fisicamente que, se as cargas podem se mover livremente, a localização na superfície - fenômeno que ocorre em condutores carregados com configurações de cargas estáticas - é tal que minimiza a energia eletrostática.
\end{exercício}
\begin{proof}[Resolução]
    Pelo \cref{ex:exercício4}, o campo elétrico para o caso da esfera de raio \(R\) maciça isolante uniformemente carregada com carga \(Q\) é dado por
    \begin{equation*}
        \vetor{E}_\mathrm{isolante}(\vetor{\x}) = \begin{cases}
            \dfrac{Q}{4\pi \epsilon_0 R^3}\vetor{\x}, &\text{ se }0 < \norm{\vetor{\x}} \leq R\\
            \dfrac{Q}{4\pi\epsilon_0\norm{\vetor{\x}}^3}\vetor{\x}, &\text{ se }\norm{\vetor{\x}} > R,
        \end{cases}
    \end{equation*}
    portanto a densidade de energia para esta configuração é
    \begin{equation*}
        u_\mathrm{isolante}(\vetor{\x}) = \frac12 \epsilon_0 \norm{\vetor{E}_\mathrm{isolante}(\vetor{\x})}^2= \begin{cases}
            \dfrac{Q^2\norm{\x}^2}{32\pi^2 \epsilon_0 R^6}, &\text{ se }0 < \norm{\vetor{\x}} \leq R\\
            \dfrac{Q^2}{32\pi^2\epsilon_0\norm{\vetor{\x}}^4}, &\text{ se }\norm{\vetor{\x}} > R.
        \end{cases}
    \end{equation*}
    Integrando, obtemos a energia eletrostática armazenada nesta configuração,
    \begin{align*}
        U_\mathrm{isolante} = \int_{\mathbb{R}^3}\dln3\x u_\mathrm{isolante}(\vetor{\x}) = \frac{Q^2}{8\pi \epsilon_0} \left(\frac{1}{R^6}\int_0^R \dli{r} r^4 + \int_R^\infty \dli{r} r^{-2}\right) = \frac{3Q^2}{20\pi \epsilon_0R}.
    \end{align*}

    Façamos o mesmo para uma esfera condutora de raio \(R\) e carregada com carga \(Q\). Como a superfície do condutor é o único lugar com presença de cargas, temos
    \begin{equation*}
        \vetor{E}_\mathrm{condutor}(\vetor{\x}) = \begin{cases}
            \vetor{0},&\text{se }0 < \norm{\vetor{\x}} < R\\
            \dfrac{Q}{4\pi \epsilon_0 \norm{\vetor{\x}}^3}\vetor{x},&\text{se }\norm{\vetor{\x}} > R
        \end{cases}
    \end{equation*}
    pela lei de Gauss. Assim, obtemos
    \begin{equation*}
        U_\mathrm{condutor} = \frac12 \epsilon_0 \int_{\mathbb{R}^3}\dln3\x \norm{\vetor{E}_\mathrm{condutor}(\vetor{\x})}^2 = \frac{Q^2}{8\pi \epsilon_0} \int_R^\infty \dli{r} r^{-2} = \frac{Q^2}{8\pi \epsilon_0 R}.
   \end{equation*}
   para a energia eletrostática armazenada nesta configuração.
\end{proof}
