\begin{exercício}{}{exercício13}
    Suponha que, ao invés da Lei de Coulomb, novos experimento extremamente precisos relevaram que força elétrica entre duas cargas pontuais \(q_1\) e \(q_2\) em repouso é, de fato;
    \begin{equation*}
        \vetor{F}_{1(2)} = \left(1 - \sqrt{a \norm{\D}}\right)\frac{q_1q_2}{4\pi \epsilon_0} \frac{\D}{\norm{\D}^3},
    \end{equation*}
    em que \(\alpha > 0\) e \(\vetor{F}_{2(1)} = -\vetor{F}_{1(2)}\). A tarefa aqui é reconstruir as Leis fundamentais da Eletrostática, supondo que ainda é válido o Princípio da Superposição.
    \begin{enumerate}[label=(\alph*)]
        \item Comece definindo uma expressão para o campo elétrico \(\vetor{E}\) de uma carga pontual \(q\) e, a partir disso, obtenha uma expressão integral para \(\vetor{E}\) de uma distribuição contínua de cargas \(\rho(\vetor{\x})\).
        \item Calcule \(\nabla \cdot \vetor{E}\) diretamente a partir da expressão integral obtida no item (a). Compare com o que deveria ser a partir da Lei de Coulomb.
        \item Considere uma carga pontual e um plano contendo essa carga (para facilitar, você pode colocar a carga na origem e considerar o plano \(xy\)). Escolha uma curva fechada \(\Gamma\) sobre o plano em torno dessa carga e calcule \(\oint_\Gamma \vetor{E}\cdot \dl{\vetor{\ell}}\). Compare com o resultado obtido a partir da Lei de Coulomb.
        \item Este novo campo elétrico admite um potencial escalar, de forma que podemos escrever \(\vetor{E} = - \nabla\phi\)? A partir disso, estude o rotacional de \(\vetor{E}\).
        \item Por fim, considere novamente uma carga pontual e calcule o fluxo \(\oint_\Sigma \vetor{E}\cdot \vetor{n}\dl{S}\), onde \(\Sigma\) é uma superfície esférica de raio \(R\) centrada na posição da carga. Compare com a Lei de Gauss da Eletrostática usual, baseada na Lei de Coulomb. Esse resultado é compatível com o que você obteve para \(\nabla\cdot \vetor{E}\) no item (b)?
    \end{enumerate}
\end{exercício}
\begin{proof}[Resolução]

\end{proof}
