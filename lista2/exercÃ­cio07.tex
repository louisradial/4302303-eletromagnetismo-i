\begin{exercício}{Campo elétrico na região de sobreposição de duas distribuições esféricas}{exercício7}
    Duas esferas, cada uma com raio \(R\) e com distribuições volumétricas de carga de densidades uniformes \(+\rho\) e \(-\rho\), respectivamente, estão posicionadas de forma que se sobrepõem parcialmente, como mostra a figura abaixo.
    \begin{center}
        \begin{tikzpicture}
            % Declare variables for coordinates and radius
            \def\xshift{1.3} % Horizontal distance between the centers of the circles
            \def\yshift{0}   % Vertical shift (if any) between the centers of the circles
            \def\yshiftlabel{0.4}
            \def\radius{2}    % Radius of the circles

            % Draw the positive charge circle (left)
            \fill[Sapphire!85,opacity=0.5] (-\xshift,\yshift) circle (\radius cm);
            \node at (-\xshift,\yshift-\yshiftlabel) {\(+\)};
            \filldraw[black] (-\xshift,\yshift) circle (2pt);

            % Draw the negative charge circle (right)
            \fill[Maroon!85,opacity=0.5] (\xshift,\yshift) circle (\radius cm);
            \node at (\xshift,\yshift-\yshiftlabel) {\(-\)};
            \filldraw[black] (\xshift,\yshift) circle (2pt);

            % Draw the arrow representing the dipole moment
            \draw[->,thick] (-\xshift,\yshift) -- (\xshift,\yshift) node[midway, above] {\(\vetor{d}\)};
        \end{tikzpicture}
    \end{center}
    Chame o vetor que descreve o deslocamento entre os centros das esferas de \(\vetor{d}\). Calcule o campo elétrico na região de sobreposição e mostre que ele é uniforme.
\end{exercício}
\begin{proof}[Resolução]
    Seja \(\vetor{\x}_0\) a posição do ponto médio entre a reta que passa pelos centros das esferas. Pelo \cref{ex:exercício4},
    \begin{align*}
        \vetor{E}(\vetor{\x}) &= \frac{\rho}{3\epsilon_0}\left(\vetor{\x} - \vetor{\x}_0 + \frac12 \vetor{d}\right) + \frac{(-\rho)}{3 \epsilon_0}\left(\vetor{\x} - \vetor{\x}_0 - \frac12\vetor{d}\right)\\
                              &= \frac{\rho}{3 \epsilon_0}\left[\left(\vetor{\x} - \vetor{\x}_0 + \frac12 \vetor{d}\right) - \left(\vetor{\x} - \vetor{\x}_0 - \frac12 \vetor{d}\right)\right]\\
                              &= \frac{\rho}{3 \epsilon_0}\vetor{d}
    \end{align*}
    é o campo elétrico no interior da região de sobreposição.
\end{proof}
