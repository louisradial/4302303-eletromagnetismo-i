\begin{exercício}{Cabo coaxial}{exercício6}
    Um cabo coaxial longo possui uma densidade volumétrica de carga uniforme \(\rho\), positiva, no cilindro interno (raio \(a\)), e uma densidade superficial de carga uniforme na casca externa do cilindro (raio \(b\)). Essa carga superficial é negativa e de magnitude exata para que o cabo, como um todo, seja eletricamente neutro.
    \begin{center}
        \includegraphics[width=0.4\linewidth]{exercício06.png}
    \end{center}
    \begin{enumerate}[label=(\alph*)]
        \item Encontre o campo elétrico em todo o espaço (onde ele é bem definido), e escreva a densidade superficial de carga em \(s = b\) em termos de \(\rho,\) \(a\) e \(b\). Faça um gráfico da intensidade do campo elétrico em função de \(s\).
        \item Encontre agora o potencial elétrico em todo o espaço, assumindo que \(\phi(s = 0) = \phi_0\). Em particular, indique o valor da diferença de potencial \(\phi(b) - \phi(a)\).
        \item Calcule a energia eletrostática armazenada na configuração por unidade de comprimento do cabo.
\end{enumerate}
\end{exercício}
\begin{proof}[Resolução]

\end{proof}
