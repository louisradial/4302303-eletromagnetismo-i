\begin{exercício}{}{exercício11}
    Compare a energia eletrostática de uma esfera maciça isolante uniformemente carregada e uma esfera condutora, ambas de carga total \(Q\) e raio \(R\). Use essa diferença para tentar argumentar fisicamente que, se as cargas podem se mover livremente, a localização na superfície - fenômeno que ocorre em condutores carregados com configurações de cargas estáticas - é tal que minimiza a energia eletrostática.
\end{exercício}
\begin{proof}[Resolução]

\end{proof}
