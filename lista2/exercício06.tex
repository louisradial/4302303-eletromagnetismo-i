\begin{exercício}{Cabo coaxial}{exercício6}
    Um cabo coaxial longo possui uma densidade volumétrica de carga uniforme \(\rho\), positiva, no cilindro interno (raio \(a\)), e uma densidade superficial de carga uniforme na casca externa do cilindro (raio \(b\)). Essa carga superficial é negativa e de magnitude exata para que o cabo, como um todo, seja eletricamente neutro.
    \begin{center}
        \includegraphics[width=0.4\linewidth]{exercício06.png}
    \end{center}
    \begin{enumerate}[label=(\alph*)]
        \item Encontre o campo elétrico em todo o espaço (onde ele é bem definido), e escreva a densidade superficial de carga em \(s = b\) em termos de \(\rho,\) \(a\) e \(b\). Faça um gráfico da intensidade do campo elétrico em função de \(s\).
        \item Encontre agora o potencial elétrico em todo o espaço, assumindo que \(\phi(s = 0) = \phi_0\). Em particular, indique o valor da diferença de potencial \(\phi(b) - \phi(a)\).
        \item Calcule a energia eletrostática armazenada na configuração por unidade de comprimento do cabo.
\end{enumerate}
\end{exercício}
\begin{proof}[Resolução]
    Utilizemos um sistema de coordenadas cilíndricas coaxial ao eixo de simetria do cabo. Seja
    \begin{equation*}
        \varrho(\vetor{\x}) = \begin{cases}
            \rho,&\text{se }s \leq a\\
            \sigma \delta(s - b),&\text{se }s > a
        \end{cases}
    \end{equation*}
    a distribuição volumétrica de cargas do cabo coaxial, onde \(\sigma\) é a densidade superficial na casca cilíndrica de raio \(b\). Seja \(\family{\Omega_R}{R \in \mathbb{R}^+} \subset \mathbb{R}^3\) uma família de volumes cilíndricos de altura \(h\) centrados na origem e coaxiais ao cabo, indexados pelo raio \(R > 0\). A carga total \(Q(R)\) contida no cilindro \(\Omega_R\) é dada por
    \begin{align*}
        Q(R) = \int_{\Omega_R}\dln3\x \varrho(\vetor{\x}),
    \end{align*}
    logo
    \begin{equation*}
        Q(R) = \begin{cases}
            \rho \pi h R^2, &\text{se }R \leq a\\
            \rho \pi h a^2, &\text{se }a < R < b\\
            \rho \pi h a^2 + 2 \pi b h\sigma, &\text{se }R > b,
        \end{cases}
    \end{equation*}
    como facilmente se constata. Para que o cabo seja neutro, devemos ter
    \begin{equation*}
        \sigma = -\frac{a^2}{2b}\rho,
    \end{equation*}
    de modo que \(Q(R) = 0\) para todo \(R \geq b\).

    Consideremos o campo elétrico gerado pelo cabo coaxial \(\vetor{E}\). Escrevamos \(\vetor{E} = E(s, \theta, z)\vetor{u}\), onde \(E\) é a intensidade do campo elétrico e \(\vetor{u}\) é um campo vetorial adimensional unitário que fornece a direção e sentido do campo elétrico em um dado ponto, exceto nos pontos em que \(E = 0\), se houver. Note que devemos ter
    \begin{equation*}
        \diffp{E}{z} = \diffp{E}{\theta} = 0,
    \end{equation*}
    por conta da invariância por rotações ao redor de \(\vetor{e}_z\) e por translações na direção axial. Ainda, \todo[é necessário que \(\vetor{u} = \vetor{e}_s\) em todo ponto, por conta da simetria cilíndrica.] Concluímos, portanto, que o campo elétrico é da forma \(\vetor{E} = E(s) \vetor{e}_s\).

    Pela lei de Gauss, temos
    \begin{equation*}
        \oint_{\partial \Omega_R} \vetor{E}\cdot \vetor{n} \dl{S} = \frac{1}{\epsilon_0} \int_{\Omega_R}\dln3\x \varrho(\vetor{\x}) \implies 2\pi R h E(R) = \frac{Q(R)}{\epsilon_0},
    \end{equation*}
    para todo \(R > 0\) com \(R \neq b\). Desse modo,
    \begin{equation*}
        E(R) = \begin{cases}
            \dfrac{\rho R}{2 \epsilon_0},&\text{se }R \leq a\\
            \dfrac{\rho a^2}{2 \epsilon_0 R}, &\text{se }a < R < b\\
            0,&\text{se }R > b
        \end{cases}
    \end{equation*}
    descreve a intensidade do campo elétrico em todo o espaço, exceto na casca cilíndrica \(R = b\).

    \begin{figure}[!ht]
        \centering
        \begin{tikzpicture}
            \begin{axis}[
                width=0.8\linewidth,
                height=0.2\textheight,
                xmin=0, xmax=4.11,
                ymin=-0.01,ymax=2.41,
                domain=0:5,
                samples=500,
                axis lines=middle,
                xlabel={\(s\)},
                ylabel={\(\norm{\vetor{E}}\)},
                legend pos=north east,
                ytick={0,4/3,2},
                yticklabels={0,\(\frac{\rho a^2}{2 \epsilon_0 b}\), \(\frac{\rho a}{2 \epsilon_0}\)},
                xtick={0, 2, 3},
                xticklabels={0, \(a\), \(b\)},
                % grid=both,
                % grid style={line width=.1pt, draw=Surface0},
                % major grid style={line width=.2pt,draw=Overlay2},
                % minor tick num=3,
            ]
                \addplot[thick, Mauve] {x < 2 ? x : (x < 3 ? 4/x : 0)};
                \addplot[thick, dashed] {2};
                \addplot[thick, dashed] {4/3};
            \end{axis}
        \end{tikzpicture}
        \caption{Intensidade do campo elétrico gerado por um cabo coaxial infinito para todo \(s \neq b\).}
    \end{figure}

    Seja \(\family{\Gamma_r}{r \in \mathbb{R}^+}\) uma família de caminhos simples e retilíneos que começam em algum ponto tal que do eixo de simetria \(s = 0\), com direção radial, indexados pela distância ao eixo do cabo \(s = r \in \mathbb{R}^+\) de seu ponto de término. O potencial \(\phi(r)\) em algum ponto no cilindro de raio \(r\) é, portanto,
    \begin{equation*}
        \phi(r)  = \phi_0 -\int_{\Gamma_r} \vetor{E}\cdot \dl{\vetor{\ell}}.
    \end{equation*}
    Para \(r \in [0, a)\), temos
    \begin{equation*}
        \phi(r) = \phi_0 -\int_{0}^{r} \dli{s} \frac{\rho s}{2 \epsilon_0}= \phi_0 - \frac{\rho r^2}{4 \epsilon_0},
    \end{equation*}
    e em particular vale \(\phi(a) = \phi_0 - \frac{\rho a^2}{4 \epsilon_0}\), por conta da continuidade do potencial. Para \(r \in [a, b)\), temos
    \begin{equation*}
        \phi(r) = \phi(a) - \int_a^r \dli{s}\frac{\rho a^2}{2 \epsilon_0 s} = \phi_0 - \frac{\rho a^2}{4 \epsilon_0} - \frac{\rho a^2}{2 \epsilon_0} \ln\left(\frac{r}{a}\right),
    \end{equation*}
    e concluímos \(\phi(b) = \phi_0 - \frac{\rho a^2}{4 \epsilon_0} - \frac{\rho a^2}{2 \epsilon_0}\ln\left(\frac{b}{a}\right)\) por continuidade. Para \(r > b\) segue que \(\phi(r) = \phi(b)\), uma vez que o campo elétrico se anula.
\end{proof}
