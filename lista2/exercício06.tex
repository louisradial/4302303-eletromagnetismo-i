\begin{exercício}{Cabo coaxial}{exercício6}
    Um cabo coaxial longo possui uma densidade volumétrica de carga uniforme \(\rho\), positiva, no cilindro interno (raio \(a\)), e uma densidade superficial de carga uniforme na casca externa do cilindro (raio \(b\)). Essa carga superficial é negativa e de magnitude exata para que o cabo, como um todo, seja eletricamente neutro.
    \begin{center}
        \includegraphics[width=0.4\linewidth]{exercício06.png}
    \end{center}
    \begin{enumerate}[label=(\alph*)]
        \item Encontre o campo elétrico em todo o espaço (onde ele é bem definido), e escreva a densidade superficial de carga em \(s = b\) em termos de \(\rho,\) \(a\) e \(b\). Faça um gráfico da intensidade do campo elétrico em função de \(s\).
        \item Encontre agora o potencial elétrico em todo o espaço, assumindo que \(\phi(s = 0) = \phi_0\). Em particular, indique o valor da diferença de potencial \(\phi(b) - \phi(a)\).
        \item Calcule a energia eletrostática armazenada na configuração por unidade de comprimento do cabo.
\end{enumerate}
\end{exercício}
\begin{proof}[Resolução]
    Utilizemos um sistema de coordenadas cilíndricas coaxial ao eixo de simetria do cabo. Seja
    \begin{equation*}
        \varrho(\vetor{\x}) = \begin{cases}
            \rho,&\text{se }s \leq a\\
            \sigma \delta(s - b),&\text{se }s > a
        \end{cases}
    \end{equation*}
    a distribuição volumétrica de cargas do cabo coaxial, onde \(\sigma\) é a densidade superficial na casca cilíndrica de raio \(b\). Seja \(\family{\Omega_R}{R \in \mathbb{R}^+} \subset \mathbb{R}^3\) uma família de volumes cilíndricos de altura \(h\) centrados na origem e coaxiais ao cabo, indexados pelo raio \(R > 0\). A carga total \(Q(R)\) contida no cilindro \(\Omega_R\) é dada por
    \begin{align*}
        Q(R) = \int_{\Omega_R}\dln3\x \varrho(\vetor{\x}),
    \end{align*}
    logo
    \begin{equation*}
        Q(R) = \begin{cases}
            \rho \pi h R^2, &\text{se }R \leq a\\
            \rho \pi h a^2, &\text{se }a < R < b\\
            \rho \pi h a^2 + 2 \pi b h\sigma, &\text{se }R > b,
        \end{cases}
    \end{equation*}
    como facilmente se constata. Para que o cabo seja neutro, devemos ter
    \begin{equation*}
        \sigma = -\frac{a^2}{2b}\rho,
    \end{equation*}
    de modo que \(Q(R) = 0\) para todo \(R \geq b\).

    Consideremos o potencial eletrostático \(\phi = \phi(s, \theta, z)\). Pela simetria de translação na direção do eixo do cabo e pela rotação ao redor deste eixo devemos ter
    \begin{equation*}
        \diffp{\phi}{z} = \diffp{\phi}{\theta} = 0,
    \end{equation*}
    donde segue que \(\phi = \phi(s)\). Concluímos, portanto, que o campo elétrico é dado por \(\vetor{E} = -\diffp{\phi}{s} \vetor{e}_s\), e podemos escrever \(\vetor{E} = E(s) \vetor{e}_s,\) onde \(E\) é a intensidade do campo.

    Pela lei de Gauss, temos
    \begin{equation*}
        \oint_{\partial \Omega_R} \vetor{E}\cdot \vetor{n} \dl{S} = \frac{1}{\epsilon_0} \int_{\Omega_R}\dln3\x \varrho(\vetor{\x}) \implies 2\pi R h E(R) = \frac{Q(R)}{\epsilon_0},
    \end{equation*}
    para todo \(R > 0\) com \(R \neq b\). Desse modo,
    \begin{equation*}
        E(R) = \begin{cases}
            \dfrac{\rho R}{2 \epsilon_0},&\text{se }R \leq a\\
            \dfrac{\rho a^2}{2 \epsilon_0 R}, &\text{se }a < R < b\\
            0,&\text{se }R > b
        \end{cases}
    \end{equation*}
    descreve a intensidade do campo elétrico em todo o espaço, exceto na casca cilíndrica \(R = b\). Isto é, o campo elétrico é dado por
    \begin{equation*}
        \vetor{E}(s \vetor{e}_s + z \vetor{e}_z) = \begin{cases}
            \dfrac{\rho s}{2 \epsilon_0}\vetor{e}_s,&\text{se }s \leq a\\
            \dfrac{\rho a^2}{2 \epsilon_0 s}\vetor{e}_s, &\text{se }a < s < b\\
            \vetor{0},&\text{se }s > b.
        \end{cases}
    \end{equation*}

    \begin{figure}[!ht]
        \centering
        \begin{tikzpicture}
            \begin{axis}[
                width=0.8\linewidth,
                height=0.2\textheight,
                xmin=0, xmax=4.1,
                ymin=-0.01,ymax=1.25,
                domain=0:5,
                samples=10000,
                axis lines=middle,
                xlabel={\(s\)},
                ylabel={\(\norm{\vetor{E}}\)},
                legend pos=north east,
                ytick={0,1/e,1},
                yticklabels={0,\(\frac{\rho a^2}{2 \epsilon_0 b}\), \(\frac{\rho a}{2 \epsilon_0}\)},
                xtick={0,1,e},
                xticklabels={0, \(a\), \(b\)},
                % grid=both,
                % grid style={line width=.1pt, draw=Surface0},
                % major grid style={line width=.2pt,draw=Overlay2},
                % minor tick num=3,
            ]
                \addplot[very thick, Mauve, jump mark left] {x < 1 ? x : (x < e ? 1/x : 0)};
                \draw[dotted] (axis cs:1,1) -- (axis cs:1,0);
                \draw[dashed, thick, Mauve] (axis cs:e,1/e) -- (axis cs:e,0);
                \addplot[dashed] {1};
                \addplot[dashed] {1/e};
            \end{axis}
        \end{tikzpicture}
        \caption{Intensidade do campo elétrico gerado por um cabo coaxial infinito para todo \(s \neq b\).}
    \end{figure}

    Seja \(\family{\Gamma_r}{r \in \mathbb{R}^+}\) uma família de caminhos simples e retilíneos que começam em algum ponto tal que do eixo de simetria \(s = 0\), com direção radial, indexados pela distância ao eixo do cabo \(s = r \in \mathbb{R}^+\) de seu ponto de término. O potencial \(\phi(r)\) em algum ponto no cilindro de raio \(r\) é, portanto,
    \begin{equation*}
        \phi(r)  = \phi_0 -\int_{\Gamma_r} \vetor{E}\cdot \dl{\vetor{\ell}}.
    \end{equation*}
    Para \(r \in [0, a)\), temos
    \begin{equation*}
        \phi(r) = \phi_0 -\int_{0}^{r} \dli{s} \frac{\rho s}{2 \epsilon_0}= \phi_0 - \frac{\rho r^2}{4 \epsilon_0},
    \end{equation*}
    e em particular vale \(\phi(a) = \phi_0 - \frac{\rho a^2}{4 \epsilon_0}\), por conta da continuidade do potencial. Para \(r \in [a, b)\), temos
    \begin{equation*}
        \phi(r) = \phi(a) - \int_a^r \dli{s}\frac{\rho a^2}{2 \epsilon_0 s} = \phi_0 - \frac{\rho a^2}{4 \epsilon_0} - \frac{\rho a^2}{2 \epsilon_0} \ln\left(\frac{r}{a}\right),
    \end{equation*}
    e concluímos \(\phi(b) = \phi_0 - \frac{\rho a^2}{4 \epsilon_0} - \frac{\rho a^2}{2 \epsilon_0}\ln\left(\frac{b}{a}\right)\) por continuidade. Para \(r > b\) segue que \(\phi(r) = \phi(b)\), uma vez que o campo elétrico se anula.

    \begin{figure}[!ht]
        \centering
        \begin{tikzpicture}
            \begin{axis}[
                width=0.8\linewidth,
                height=0.2\textheight,
                xmin=0, xmax=4.1,
                ymin=-0.01,ymax=1.85,
                domain=0:5,
                samples=10000,
                axis lines=middle,
                xlabel={\(s\)},
                ylabel={\phi(s) - \phi(b)},
                legend pos=north east,
                ytick={0,1,3/2},
                yticklabels={0,\(\frac{\rho a^2}{2 \epsilon_0}\ln\left(\frac{b}{a}\right)\), \(\frac{\rho^2}{4 \epsilon_0} + \frac{\rho a^2}{2 \epsilon_0}\ln\left(\frac{b}{a}\right)\)},
                xtick={0,1,e},
                xticklabels={0, \(a\), \(b\)},
                % grid=both,
                % grid style={line width=.1pt, draw=Surface0},
                % major grid style={line width=.2pt,draw=Overlay2},
                % minor tick num=3,
            ]
                \addplot[very thick, Mauve, jump mark left] {x < 1 ? 3/2 - x^2/2 : (x < e ? 1 - ln(x) : 0)};
                \draw[dotted, Mauve] (axis cs:1,1) -- (axis cs:1,0);
                \addplot[dashed] {3/2};
                \addplot[dashed] {1};
            \end{axis}
        \end{tikzpicture}
        \caption{Diferença de potencial elétrico em relação a \(\phi(b)\).}
    \end{figure}
    Isto é, o potencial é dado por
    \begin{equation*}
        \phi(s) = \begin{cases}
            \phi_0 - \dfrac{\rho s^2}{4 \epsilon_0}, &\text{se }s \in [0,a)\\
            \phi_0 - \dfrac{\rho a^2}{2 \epsilon_0}\left[\dfrac{1}{2} + \ln\left(\dfrac{s}{a}\right)\right], &\text{se }s\in[a,b]\\
            \phi_0 - \dfrac{\rho a^2}{2 \epsilon_0}\left[\dfrac{1}{2} + \ln\left(\dfrac{b}{a}\right)\right], &\text{se }s\in(b, \infty)
        \end{cases}
    \end{equation*}
    e em particular temos \(\phi(b) - \phi(a) = - \frac{\rho a^2}{2 \epsilon_0} \ln\left(\frac{b}{a}\right)\).

    A densidade de energia eletrostática armazenada na configuração é dada por
    \begin{equation*}
        u = \frac12 \epsilon_0 \norm{E}^2 = \begin{cases}
            \frac{\rho^2 s^2}{8\epsilon_0}, & \text{se }s \in [0, a]\\
            \frac{\rho^2 a^4}{8 \epsilon_0 s^2}, &\text{se }s \in (a, b)\\
            0, &\text{se }s \in (b, \infty).
        \end{cases}
    \end{equation*}
    Desse modo, a energia \(U_{\ell}\) armazenada em um volume \(\Sigma_\ell = \setc*{\vetor{\x} \in \mathbb{R}^3}{\inner{\vetor{e}_z}{\vetor{\x}} \in [0, \ell]}\subset \mathbb{R}^3\) é
    \begin{equation*}
        U_{\ell} = \int_{\Sigma_\ell}\dln3x u = 2\pi \ell \int_{0}^\infty \dli{s} su = \frac{2\pi \ell \rho^2}{8 \epsilon_0}\left( \int_0^a \dli{s} s^3 + \int_a^b \dli{s} a^4s^{-1}\right) = \frac{\pi \ell \rho^2a^4}{4 \epsilon_0}\left[\frac{1}{4} + \ln\left(\frac{b}{a}\right)\right].
    \end{equation*}
    Concluímos, portanto, que
    \begin{equation*}
        \frac{U_\ell}{\ell} = \frac{\pi\rho^2a^4}{4 \epsilon_0}\left[\frac{1}{4} + \ln\left(\frac{b}{a}\right)\right]
    \end{equation*}
    é a energia eletrostática armazenada na configuração por unidade de comprimento do cabo.
\end{proof}
