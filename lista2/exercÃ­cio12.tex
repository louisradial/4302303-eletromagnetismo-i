\begin{exercício}{}{exercício12}
    Considere um condutor carregado cujo formato é arbitrário. A densidade de carga sobre os pontos de sua superfície é \(\sigma(\vetor{\x})\). Mostre que a força elétrica por unidade de área sobre um elemento de carga localizado em \(\vetor{\x}\), devido a todo o restante da superfície, é dado por
    \begin{equation*}
        \vetor{f}(\vetor{x}) = \frac{\sigma(\vetor{\x})^2}{2 \epsilon_0}\vetor{n},
    \end{equation*}
    onde \(\vetor{n}\) é a normal à superfície no ponto \(\vetor{\x}\).
\end{exercício}
\begin{proof}[Resolução]

\end{proof}
