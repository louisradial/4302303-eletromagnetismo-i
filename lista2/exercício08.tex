\begin{exercício}{}{exercício8}
    Uma distribuição de carga estática produz um campo elétrico radial da forma
    \begin{equation*}
        \vetor{E}(\vetor{\x}) = A\frac{\exp(-br)}{r^2}\vetor{e}_r,
    \end{equation*}
    em que \(A \neq 0\) e \(b > 0\) são constantes. Encontre a densidade de carga \(\rho\) e faça um gráfico de \(\rho\) em função de \(r\). Qual é a carga total desse sistema?
\end{exercício}
\begin{proof}[Resolução]
    Temos
    \begin{align*}
        \nabla \cdot \vetor{E} &= A \exp(-br)\nabla\left(\frac1r^2\vetor{e}_r\right) + A \left(\frac{1}{r^2}\vetor{e}_r\right)\nabla[\exp(-br)]\\
        &= 4\pi A \delta(r) \exp(-br) - \frac{bA\exp(-br)}{r^2}\\
        &= \left[4\pi \delta(\vetor{\x}) - \frac{b \exp(-b\norm{\vetor{\x}})}{\norm{\vetor{\x}}^2}\right]A
    \end{align*}
    como a expressão do divergente do campo elétrico. Assim, pela Lei de Gauss, a densidade de carga é dada por
    \begin{equation*}
        \rho(\vetor{\x}) = \left[4\pi \delta(\vetor{\x}) - \frac{b \exp(-b\norm{\vetor{\x}})}{\norm{\vetor{\x}}^2}\right]\epsilon_0 A.
    \end{equation*}

    \begin{figure}[!h]
        \centering
        \begin{tikzpicture}
            \begin{axis}[
                width=0.8\linewidth,
                height=0.25\textheight,
                xmin=0.2, xmax=2,
                ymin=-4,ymax=0.5,
                domain=0.2:2,
                samples=500,
                axis lines=middle,
                xlabel={$r$},
                ylabel={$\rho(r)$},
                legend pos=north east,
                ytick=\empty,
                xtick=\empty
            ]
                \addplot[thick, Mauve] {-0.2*exp(-0.2*x)/x^2};
            \end{axis}
        \end{tikzpicture}
        \caption{Densidade de carga do \cref{ex:exercício8} para \(r > 0\).}
    \end{figure}

    Notemos que
    \begin{align*}
        Q = \int_{\mathbb{R}^3} \dln3\x \rho(\vetor{\x}) = \left[4\pi - 4\pi b \int_{0}^\infty \dli{r} \exp(-br)\right]\epsilon_0 A = 0,
    \end{align*}
    isto é, a carga total do sistema se anula.
\end{proof}
