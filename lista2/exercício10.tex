\begin{exercício}{}{exercício10}
    Considere duas cargas pontuais \(q_1\) e \(q_2\), localizadas nos pontos \(\vetor{\x}_1\) e \(\vetor{\x}_2\). Não há mais nenhuma outra carga no espaço. Obtenha a energia eletrostática associada à interação entre as cargas, \(W\), relacionando-a com o trabalho realizado para construir a configuração.

    A seguir, nosso objetivo é calcular a energia total armazenada na configuração através da expressão geral
    \begin{equation*}
        U_E = \frac12 \epsilon_0 \int\dln3x\norm{\vetor{E}}^2
    \end{equation*}
    onde \(\vetor{E} = \vetor{E}_1 + \vetor{E}_2\), e compará-la com o seu cálculo anterior. Você deve lembrar das aulas que essa integral não converge para campos de cargas pontuais. Porém, antes de avaliar a integral explicitamente, mostre que \(U_E\) pode ser reorganizado como
    \begin{equation*}
        U_E = U_1 + U_2 + U_\mathrm{int},
    \end{equation*}
    em que \(U_1\) e \(U_2\) são as energias eletrostáticas associadas apenas aos campos \(\vetor{E}_1\) e \(\vetor{E}_2\), individualmente, e \(U_{\mathrm{int}}\) é um termo de interação envolvendo produtos cruzados entre os dois campos. Prove que, ao contrário de \(U_1\) e \(U_2\), \(U_\mathrm{int}\) converge e é exatamente igual à \(W\). Interprete fisicamente.
\end{exercício}
\begin{proof}[Resolução]

\end{proof}
