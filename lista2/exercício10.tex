\begin{exercício}{Energia de interação e energia armazenada na configuração}{exercício10}
    Considere duas cargas pontuais \(q_1\) e \(q_2\), localizadas nos pontos \(\vetor{\x}_1\) e \(\vetor{\x}_2\). Não há mais nenhuma outra carga no espaço. Obtenha a energia eletrostática associada à interação entre as cargas, \(W\), relacionando-a com o trabalho realizado para construir a configuração.

    A seguir, nosso objetivo é calcular a energia total armazenada na configuração através da expressão geral
    \begin{equation*}
        U_E = \frac12 \epsilon_0 \int_{\mathbb{R}^3}\dln3x\norm{\vetor{E}}^2
    \end{equation*}
    onde \(\vetor{E} = \vetor{E}_1 + \vetor{E}_2\), e compará-la com o seu cálculo anterior. Você deve lembrar das aulas que essa integral não converge para campos de cargas pontuais. Porém, antes de avaliar a integral explicitamente, mostre que \(U_E\) pode ser reorganizado como
    \begin{equation*}
        U_E = U_1 + U_2 + U_\mathrm{int},
    \end{equation*}
    em que \(U_1\) e \(U_2\) são as energias eletrostáticas associadas apenas aos campos \(\vetor{E}_1\) e \(\vetor{E}_2\), individualmente, e \(U_{\mathrm{int}}\) é um termo de interação envolvendo produtos cruzados entre os dois campos. Prove que, ao contrário de \(U_1\) e \(U_2\), \(U_\mathrm{int}\) converge e é exatamente igual à \(W\). Interprete fisicamente.
\end{exercício}
\begin{proof}[Resolução]
    Dada uma carga \(q_1\) na posição \(\vetor{\x}_1\), o trabalho necessário para colocar a carga \(q_2\) na posição \(\vetor{\x}_2\) é dado por
    \begin{equation*}
        W = - q_2 \int_\infty^{\vetor{\x}_2} \vetor{E}_1 \cdot \dl{\vetor{\ell}} = \frac{q_1q_2}{4\pi \epsilon_0 \norm{\vetor{\x}_1 - \vetor{\x}_2}},
    \end{equation*}
    notando que o mesmo resultado seria obtido se começássemos com a carga \(q_2\) na posição \(\vetor{\x}_2\) e desejássemos colocar a carga \(q_1\) na posição \(\vetor{\x}_1\).

    Consideremos agora a densidade de energia da configuração,
    \begin{equation*}
        u = \frac12 \epsilon_0 \norm{\vetor{E}}^2,
    \end{equation*}
    onde o campo elétrico é dada pela superposição \(\vetor{E} = \vetor{E}_1 + \vetor{E}_2\). Assim, sendo
    \begin{equation*}
        u_1 = \frac12 \epsilon_0 \norm{\vetor{E}_1}^2,\quad
        u_2 = \frac12 \epsilon_0 \norm{\vetor{E}_2}^2,\quad\text{e}\quad
        u_\mathrm{int} = \epsilon_0 \vetor{E}_1\cdot\vetor{E}_2
    \end{equation*}
    as densidades de auto-energia de \(\vetor{E}_1\) e de \(\vetor{E}_2\) e densidade de energia interação entre \(\vetor{E}_1\) e \(\vetor{E}_2\), respectivamente, temos \(u = u_1 + u_2 + u_\mathrm{int}\).

    % % Para um dia de chuva %
    % A densidade de energia de interação é dada por
    % \begin{equation*}
    %     u_\mathrm{int}(\vetor{\x}) = \frac{q_1q_2}{16\pi^2 \epsilon_0} \frac{\inner{\vetor{\x} - \vetor{\x}_1}{\vetor{\x} - \vetor{\x}_2}}{\norm{\vetor{\x}- \vetor{\x}_1}^3 \norm{\vetor{\x} - \vetor{\x}_2}^3},
    % \end{equation*}
    % portanto definindo \(\vetor{\x}_0 = \frac12 (\vetor{\x}_1 + \vetor{\x}_2)\) e \(\vetor{\xi} = \frac12 (\vetor{\x}_1 - \vetor{\x}_2)\), temos
    % \begin{equation*}
    %     u_\mathrm{int}(\vetor{\x}) = \frac{q_1q_2}{16\pi^2 \epsilon_0}\frac{\norm{\vetor{\x} - \vetor{\x}_0}^2 - \norm{\vetor{\xi}}^2}{\norm{\vetor{\x} - \vetor{\x}_0 + \vetor{\xi}}^3\norm{\vetor{\x} - \vetor{\x}_0 - \vetor{\xi}}^3}.
    % \end{equation*}

    Notemos que
    \begin{equation*}
        \vetor{E}_1\cdot\vetor{E}_2 = \phi_2 \nabla \cdot \vetor{E}_1 - \nabla \cdot (\phi_2 \vetor{E}_1),
    \end{equation*}
    onde \(\phi_2\) é o potencial eletrostático considerando apenas a carga \(q_2\),
    \begin{equation*}
        \phi_2(\vetor{\x}) = \frac{q_2}{4\pi \epsilon_0 \norm{\vetor{\x} - \vetor{\x}_2}}.
    \end{equation*}
    Seja \(\Omega_R\) um volume esférico de raio \(R > 0\) com \(R \gg \norm{\vetor{\x}_1 - \vetor{\x}_2}\), então a energia de interação contida em \(\Omega_R\) é
    \begin{equation*}
        U_\mathrm{int}(\Omega_R) = \int_{\Omega_R} \dln3\x u_{\mathrm{int}}(\vetor{\x}) = \int_{\Omega_R}\dln3\x \phi_2(\vetor{\x}) q_1 \delta(\vetor{\x} - \vetor{\x}_1) - \int_{\partial \Omega_R} \dli{a} \vetor{n} \cdot \phi_2(\vetor{\x})\vetor{E}_1(\vetor{\x}).
    \end{equation*}
    Notemos que \(\phi_2 \vetor{E}_1\cdot\vetor{n}\simeq R^{-3}\), portanto quando tomamos o limite \(R \to \infty\), obtemos
    \begin{equation*}
        U_\mathrm{int} = q_1 \phi_2(\vetor{\x}_1) = \frac{q_1q_2}{4\pi \epsilon_0 \norm{\vetor{\x_1} - \vetor{\x}_2}},
    \end{equation*}
    isto é, \(U_\mathrm{int} = W\).
\end{proof}
