\begin{exercício}{Densidade de carga para o estado fundamental do átomo de hidrogênio}{exercício9}
    O potencial eletrostático médio criado pelo átomo de hidrogênio em seu estado fundamental pode ser escrito em coordenadas esféricas como
    \begin{equation*}
        \phi(r, \theta, \varphi) = \frac{q_e}{4\pi \epsilon_0} \frac{\exp\left(-\frac{2r}{a_0}\right)}{r} \left(1 + \frac{r}{a_0}\right),
    \end{equation*}
    sendo \(q_e\) a carga elementar e \(a_0\) o raio de Bohr. Obtenha a densidade de carga em todo o espaço e a interprete fisicamente.
\end{exercício}
\begin{proof}[Resolução]

\end{proof}
