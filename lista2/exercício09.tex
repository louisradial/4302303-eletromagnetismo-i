\begin{exercício}{Densidade de carga para o estado fundamental do átomo de hidrogênio}{exercício9}
    O potencial eletrostático médio criado pelo átomo de hidrogênio em seu estado fundamental pode ser escrito em coordenadas esféricas como
    \begin{equation*}
        \phi(r, \theta, \varphi) = \frac{q_e}{4\pi \epsilon_0} \frac{\exp\left(-\frac{2r}{a_0}\right)}{r} \left(1 + \frac{r}{a_0}\right),
    \end{equation*}
    sendo \(q_e\) a carga elementar e \(a_0\) o raio de Bohr. Obtenha a densidade de carga em todo o espaço e a interprete fisicamente.
\end{exercício}
\begin{proof}[Resolução]
    Como o potencial eletrostático médio depende apenas da distância radial, temos
    \begin{align*}
        \nabla^2\phi = \frac{1}{r}\diffp*[2]{(r\phi)}{r}
        &= \frac{q_e}{4\pi \epsilon_0r}\diffp*{ \left[\frac{1}{a_0}\exp\left(-\frac{2r}{a_0}\right) - \frac{2}{a_0}\left(1 + \frac{r}{a_0}\right)\exp\left(-\frac{2r}{a_0}\right)\right]}{r}\\
        &= -\frac{q_e}{4\pi \epsilon_0a_0r}\diffp*{ \left[\left(1 + \frac{2r}{a_0}\right)\exp\left(-\frac{2r}{a_0}\right)\right]}{r}\\
        &= - \frac{q_e}{4\pi \epsilon_0 a_0 r}\left[\frac{2}{a_0}\exp\left(-\frac{2r}{a_0}\right) - \frac{2}{a_0}\left(1 + \frac{r}{a_0}\right)\exp\left(-\frac{2r}{a_0}\right)\right]\\
        &= \frac{q_e}{2\pi \epsilon_0 a_0^3}\exp\left(-\frac{2r}{a_0}\right).
    \end{align*}
    Da equação de Poisson, segue que
    \begin{equation*}
        \rho(r, \theta, \varphi) = \frac{q_e}{2\pi a_0^3} \exp\left(-\frac{2r}{a_0}\right)
    \end{equation*}
    é a densidade de carga.
\end{proof}
