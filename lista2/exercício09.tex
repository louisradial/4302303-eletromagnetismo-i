\begin{exercício}{Densidade de carga para o estado fundamental do átomo de hidrogênio}{exercício9}
    O potencial eletrostático médio criado pelo átomo de hidrogênio em seu estado fundamental pode ser escrito em coordenadas esféricas como
    \begin{equation*}
        \phi(r, \theta, \varphi) = \frac{q_e}{4\pi \epsilon_0} \frac{\exp\left(-\frac{2r}{a_0}\right)}{r} \left(1 + \frac{r}{a_0}\right),
    \end{equation*}
    sendo \(q_e\) a carga elementar e \(a_0\) o raio de Bohr. Obtenha a densidade de carga em todo o espaço e a interprete fisicamente.
\end{exercício}
\begin{proof}[Resolução]
    A regra de Leibniz fornece
    \begin{align*}
        \diffp{\phi}{r} &= \frac{q_e}{4\pi \epsilon_0} \left[-\frac{2}{a_0r}\left(1 + \frac{r}{a_0}\right)\exp\left(-\frac{2r}{a_0}\right) - \frac{1}{r^2}\left(1 + \frac{r}{a_0}\right)\exp\left(-\frac{2r}{a_0}\right) + \frac{1}{a_0 r}\exp\left(-\frac{2r}{a_0}\right)\right]\\
                        &= \frac{q_e}{4\pi \epsilon_0 a_0^2 r^2}\left[-2\left(a_0r + r^2\right)-(a_0^2 + a_0r)+ a_0r\right]\exp\left(-\frac{2r}{a_0}\right)\\
                        &= -\frac{q_e}{4\pi \epsilon_0 a_0^2 r^2}\left(2r^2 + 2a_0 r + a_0^2\right)\exp\left(-\frac{2r}{a_0}\right),
    \end{align*}
    portanto como o potencial eletrostático médio depende apenas da distância radial, temos
    \begin{equation*}
        \vetor{E}(r \vetor{e}_r) = -\nabla\phi = \frac{q_e}{4\pi \epsilon_0 a_0^2r^2}\left(2r^2 + 2a_0 r + a_0^2\right)\exp\left(-\frac{2r}{a_0}\right)\vetor{e}_r
    \end{equation*}
    como a expressão para o campo elétrico.

    Seja \(f(r) = \left(2r^2 + 2a_0 r + a_0^2\right)\exp\left(-\frac{2r}{a_0}\right)\), de modo que \(\vetor{E} \propto \frac{f(r)}{r^2} \vetor{e}_r\). Notemos que
    \begin{equation*}
        \nabla f = \frac{(4a_0r + 2a_0^2) - 2 (2r^2 + 2a_0r + a_0^2)}{a_0}\exp\left(-\frac{2r}{a_0}\right)\vetor{e}_r = -\frac{4r^2}{a_0} \exp\left(-\frac{2r}{a_0}\right)\vetor{e}_r
    \end{equation*}
    donde segue
    \begin{align*}
        \nabla\cdot\vetor{E} &= \frac{q_e}{4\pi \epsilon_0 a_0^2} \left[\left(\frac{1}{r^2}\vetor{e}_r\right)\cdot \nabla f + f \nabla \cdot \left(\frac{1}{r^2}\vetor{e}_r\right)\right]\\
                             &= \frac{q_e}{4\pi \epsilon_0 a_0^2} \left[-\frac{4}{a_0}\exp\left(-\frac{2r}{a_0}\right) + 4\pi \delta(r)\left(2r^2 + 2a_0 r + a_0^2\right)\exp\left(-\frac{2r}{a_0}\right)\right]\\
                             &= \frac{q_e}{\epsilon_0}\left[\delta(r) - \frac{1}{\pi a_0^3}\exp\left(-\frac{2r}{a_0}\right)\right].
    \end{align*}
    Assim, pela lei de Gauss, segue que
    \begin{equation*}
        \rho(\vetor{\x}) = q_e\left[\delta(\vetor{\x}) - \frac{1}{\pi a_0^3}\exp\left(-\frac{2\norm{\vetor{\x}}}{a_0}\right)\right]
    \end{equation*}
    é a densidade de carga em todo o espaço.

    Esta densidade de carga corresponde a uma carga \(+q_e\) pontual localizada na origem do sistema de coordenadas e uma distribuição isotrópica de carga \(\rho_e(\vetor{\x}) = \rho(\vetor{\x}) - q_e \delta(\vetor{\x})\) cuja carga total \(Q_e(R)\) contida em uma esfera de raio \(R\) é
    \begin{align*}
        Q_e(R) &= -\frac{q_e}{\pi a_0^3}\int_0^R \dli{r} \int_0^\pi r \dli{\theta} \int_0^{2\pi} r \sin\theta \dli{\varphi} \exp\left(-\frac{2r}{a_0}\right)\\
          &= -\frac{q_e}{a_0^2} \int_0^R \dli{r} \frac{4r^2}{a_0} \exp\left(-\frac{2r}{a_0}\right)\\
          &= \frac{q_e}{a_0^2} \left[f(R) - f(0)\right]\\
          &= q_e\left[\frac{1}{2}\left(\frac{2R}{a_0}\right)^2 + \frac{2R}{a_0} + 1\right]\exp\left(-\frac{2R}{a_0}\right) - q_e.
    \end{align*}
    \begin{figure}[!ht]
        \centering
        \begin{tikzpicture}
            \begin{axis}[
                width=0.8\linewidth,
                height=0.25\textheight,
                xmin=0, xmax=15.1,
                ymin=0,ymax=1.22,
                domain=0:15.1,
                samples=500,
                axis lines=middle,
                xlabel={\(R\)},
                ylabel={\(-Q_e(R)\)},
                legend pos=north east,
                ytick={0, 0.25, 0.5, 0.75, 1, 1.25},
                yticklabels={0, \(\frac{1}{4}q_e\), \(\frac{1}{2}q_e\), \(\frac{3}{4}q_e\), \(q_e\)},
                xtick={0,2,4,6,8,10,12,14,16},
                xticklabels={0, \(a_0\), \(2a_0\), \(3a_0\), \(4a_0\), \(5a_0\), \(6a_0\), \(7a_0\),},
                grid=both,
                grid style={line width=.1pt, draw=Surface0},
                major grid style={line width=.2pt,draw=Overlay2},
                minor tick num=3,
            ]
                \addplot[thick, Mauve] {1 - (0.5*x^2 + x + 1)*exp(-x)};
                \addplot[thick, dashed] {1};
            \end{axis}
        \end{tikzpicture}
        \caption{Carga \(Q_e(R)\) contida em uma esfera de raio \(R\) devido à distribuição isotrópica \(\rho_e(\vetor{\x})\).}
    \end{figure}

    No limite em que \(R \to \infty\), temos que \(Q_e(R)\) tende a \(-q_e\), de modo que a carga total de \(\rho\) em todo espaço se anula. A densidade \(\rho_e\) decresce exponencialmente, ao ponto que \(Q(6 a_0) \simeq -q_e\) com erro de 0.05\%. Ainda, temos
    \begin{align*}
        \langle \norm{\vetor{\x}_e} \rangle = \frac{1}{-q_e} \int_{\mathbb{R}^3} \dln3\x \norm{\vetor{\x}} \rho_e(\vetor{\x})
        &= \int_0^\infty \dli{r} \frac{4r^3}{a_0^3}\exp\left(-\frac{2r}{a_0}\right)\\
        &= \frac14a_0 \int_0^\infty \dli\xi \xi^3 e^{-\xi}\\
        &= -\frac14a_0 \diffp*[3]{\left(\int_0^\infty \dli\xi e^{-\beta \xi}\right)}{\beta}[\beta = 1]\\
        &= -\frac14a_0 \diff*[3]{\beta^{-1}}{\beta}[\beta = 1]\\
        &= \frac32a_0,
    \end{align*}
    de forma que podemos interpretar a a distribuição de carga \(\rho_e\) como uma partícula de carga \(-q_e\) não localizada mas com valor esperado de sua distância à carga \(+q_e\) igual a \(\frac32 a_0\). Concluímos, portanto, que as densidades de carga \(q_e \delta(\vetor{\x})\) e \(\rho_e\) correspondem ao próton do núcleo do átomo de hidrogênio e ao elétron do átomo de hidrogênio.
\end{proof}
